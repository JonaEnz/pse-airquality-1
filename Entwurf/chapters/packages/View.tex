\subsection{View}
Sofern nicht anders angebeben sind alle Klassen in diesem 
\\Unterkapitel React-Komponenten.

\subsubsection{Layout}
\begin{Class}{Layout}
    Die Hauptkomponente der Anwendung. Zeigt das Seitenmenü und Anwendungsleiste an.
    Je nach Routing werden weitere Komponenten geladen.
\end{Class}

\begin{Class}{LanguageMenu}
    Eine Auswahlkomponente für Sprachen.
    Benötigt keine Attribute da die statische Language Klasse genutzt wird.
\end{Class}

\begin{Class}{AboutPage}
    Zeigt das Impressum in der gewählten Sprache an.
\end{Class}

\begin{Class}{PrivacyPolicyPage}
    Zeigt die Datenschutzerklärung in der gewählten Sprache an.
\end{Class}

\begin{Class}{ErrorPage}
    Wird bei einem kritischen Fehler aufgerufen.
    Zeigt eine Fehlermeldung an und enthält einen Button der auf die Kartenseite führt.
    \textbf{Attribute}
    \begin{itemize}
        \item \texttt{Props.errorCode}
        \\ Ein Fehlercode.
        \item \texttt{Props.errorMessage}
        \\ Der Sprachdateien-Schlüssel für die Fehlermeldung.
    \end{itemize}
\end{Class}

\subsubsection{MapPage.Components}
    \begin{Class}{MapPage}
        Die Hauptkomponente der Kartenansicht. Enthält \emph{MapView}.
    \end{Class}
    \begin{Class}{MapView}
        Eine Komponente die die Karte selbst und weitere Steuerelemente wie die Suche, Konfiguration und Legende beinhaltet.
        \textbf{Methoden}
        \begin{itemize}
            \item \texttt{selectStation(station : ObservationStation)}
            \\ Legt die angegebene ObservationStation als ausgewählt fest.
            \item \texttt{getValueAt(position: Position, feature : Feature) : float}
            \\ Berechnet mithilfe naher Messstationen einen Schätzwert des gewählten Features an einer bestimmter Position.
            \\ Rückgabe:
            \\ \emph{null}: Es konnte kein Wert bestimmt werden da der Punkt zu weit entfernt von Messstationen ist.
            \\ \emph{float}: Der geschätzte Wert
            \item \texttt{onViewportChange(viewport : Viewport)}
            \\ Verarbeitet die Änderung des Viewports und aktualisiert die Pins und Polygone auf der Karte.
            \item \texttt{onStationSelected(pin : MapPin)}
            \\ Verarbeitet den Klick auf einen Stations-Marker und aktualisiert das zugehörige Popup.
        \end{itemize}
    \end{Class}

    \begin{Class}{FeatureSelect}
        Eine Komponente für Einstellungen der Karte.
        \textbf{Methoden}
        \begin{itemize}
            \item \texttt{getSelectedFeature() : Feature}
            \\ Rückgabe:
            \\ \emph{null}: Fehler, nicht definiert
            \\ \emph{Feature}: aktuell ausgewähltes Feature
            \item \texttt{onConfigurationChange(conf: MapConfiguration)}
            \\ Wird aufgerufen wenn sich aus einer Aktion des Nutzers eine neue Kartenkonfiguration ergibt.
        \end{itemize}
    \end{Class}

    \begin{Class}{Search}
        Ein Suchfeld für Adressen und ein Button zur automatischen Positionserkennung.
        \textbf{Methoden}
        \begin{itemize}
            \item \texttt{onSearch(string searchTerm) : void}
            \\ Wird aufgerufen wenn auf den Such-Button geklickt wird.
            \\ Es wird nach der Adresse gesucht und das entsprechende Event im MapController aufgerufen.
            \item \texttt{onLocationSearch() : void}
            \\ Wird bei Klick auf den "Lokalisieren" Knopf aufgerufen und ruft die entsprechende Methode im MapController auf.
            \item \texttt{reset() : void}
            \\ Setzt searchTerm auf das leere Wort.
        \end{itemize}
        \textbf{Attribute}
        \begin{itemize}
            \item \texttt{searchTerm : string}
            \\ Der aktuelle Inhalt der Textbox der Komponente.
        \end{itemize}
    \end{Class}
    
    \begin{Class}{Map}
        Eine Leaflet-Karte mit anpassbaren Markern und Polygonen.
        \textbf{Methoden}
        \begin{itemize}
            \item \texttt{Props.onViewportChanged(viewport : Viewport)}
            \\ Wird aufgerufen wenn die Leaflet-Karte verschoben wird.
        \end{itemize}
        \textbf{Attribute}
        \begin{itemize}
            \item \texttt{Props.pins : MapPin[]}
            \\ Die Marker die auf der Karte angezeigt werden sollen.
            \item \texttt{Props.polygons : Polygon[]}
            \\ Die Polygone die auf der Karte angezeigt werden sollen.
        \end{itemize}
    \end{Class}
    
    \begin{Class}{Legend}
        Eine Legende für die Farben der aktuellen Ansicht.
        \textbf{Attribute}
        \begin{itemize}
            \item \texttt{Props.scale : Skala}
            \\ Die aktuell verwendete Skala.
        \end{itemize}
    \end{Class}
    
    \begin{Class}{StationInfo}
        Das Popup für eine bestimmte Messstation. Es beinhaltet Metadaten und einen Button zur Detailseite.
        \textbf{Attribute}
        \begin{itemize}
            \item \texttt{openDetails : MaterialUI.Button}
            \\ click() öffnet die Detailansicht mit der ID der \emph{station}
            \item \texttt{Props.station : Messstation}
            \\ Die Messstation zu der die StationInfo gehört.
        \end{itemize}
    \end{Class}


