\subsection{MapPage}

    \begin{Class}{Viewport}
        \textbf{Methoden}
        \begin{itemize}
            \item \texttt{getCenter() : Position}
            \\ Gibt die aktuelle Position des Zentrums zurück.
            \item \texttt{setCenter(position : Position) : void}
            \\ Setzt \emph{position} als das Zentrum fest.
            \item \texttt{getZoom() : float}
            \\ Liefert den Zoom des Viewports.
            \item \texttt{setZoom(zoom : float) : void}
            \\ Setzt den Zoom des Viewports auf \emph{zoom}.
            \item \texttt{getRadius() : int}
            \\ Liefert den kleinsten Radius des Kreises in Koordinateneinheiten um das Zentrum,
            der den kompletten Anzeigebereich des Viewport einschließt.
        \end{itemize}
    \end{Class}

    \begin{Class}{<<abstract>> MapConfiguration}
        \textbf{Methoden}
        \begin{itemize}
            \item \texttt{getPins(port : Viewport) : MapPin[]}
            \\ Gibt MapPins innerhalb (oder nahe) des Viewports zurück.
            \item \texttt{getPolygons(port: Viewport) : Polygon[]}
            \\ Gibt Polygone innerhalb (oder nahe) des Viewports zurück.
            \item \texttt{getScale() : Scale}
            \\ Gibt die Skala der Konfiguration zurück.
            \item \texttt{getFeatures() : Feature[]}
            \\ Gibt die durch die Konfiguration spezifizierten Features zurück.
            \\ \emph{[]}, wenn keine Features dargestellt werden
        \end{itemize}
    \end{Class}


    \begin{Class}{StationConfiguration extends MapConfiguration}
        Eine Kartenkonfiguration die die Stationen mit Farbe nach einer Skala anzeigt.
        \textbf{Methoden}
        \begin{itemize}
            \item \texttt{setFeature(features : Feature) : void}
            \\ Legt das Feature der Konfiguration fest.
            \item \texttt{<<implements>>getPins(port : Viewport) : MapPin[]}
            \\ Liefert MapPins innerhalb (oder nahe) des Viewports.
            \\ Es werden nur Stationen angezeigt die das ausgewählte Feature besitzen
            \item \texttt{<<implements>>getPolygons(port: Viewport) : Polygon[]}
            \\ Liefert [].
            \item \texttt{<<implements>>getScale() : Scale}
            \\ Liefert die Skala des gewählten Features.
            \item \texttt{<<implements>>getFeatures() : Feature[]}
            \\ Liefert das ausgewählte Feature in einem Array.
        \end{itemize}
    \end{Class}

    \begin{Class}{PolygonConfiguration extends MapConfiguration}
        Eine Kartenkonfiguration die zwischen den Stationen Polygone aufspannt und sie nach den gemittelten Werten einfärbt.
        \textbf{Methoden}
        \begin{itemize}
            \item \texttt{setFeature(feature : Feature) : void}
            \\ Legt das Feature der Konfiguration fest.
            \item \texttt{<<implements>>getPins(port : Viewport) : MapPin[]}
            \\ Liefert MapPins innerhalb (oder nahe) des Viewports.
            \\ Es werden nur Stationen angezeigt die das ausgewählte Feature besitzen
            \item \texttt{<<implements>>getPolygons(port: Viewport) : Polygon[]}
            \\ Liefert Polygone die zwischen Stationen im Viewport \emph{port} aufgespannt sind.
            \item \texttt{<<implements>>getScale() : Scale}
            \\ Liefert die Skala des gewählten Features.
            \item \texttt{<<implements>>getFeatures() : Feature[]}
            \\ Liefert das ausgewählte Feature in einem Array.
        \end{itemize}
    \end{Class}

    \begin{Class}{NearConfiguration extends MapConfiguration}
        Eine Kartenkonfiguration die die Stationen mit Farbe nach einer Skala anzeigt.
        \textbf{Methoden}
        \begin{itemize}
            \item \texttt{setFeature(feature : Feature) : void}
            \\ Legt das Feature der Konfiguration fest.
            \item \texttt{<<implements>>getPins(port : Viewport) : MapPin[]}
            \\ Liefert MapPins innerhalb des Radius \emph{radius} um die Mitte des Viewports.
            \\ Es werden nur Stationen angezeigt die das ausgewählte Feature besitzen.
            \\ Die Färbung wird anhand des Mittelwerts der Stationen im Umkreis von \emph{radius} km festgelegt.
            \\ Dabei gibt der höchste Wert die Farbe am oberen Ende der Skala an, der niedrigste das untere Ende.
            \item \texttt{<<implements>>getPolygons(port: Viewport) : Polygon[]}
            \\ Liefert immer [].
            \item \texttt{<<implements>>getScale() : Scale}
            \\ Liefert die Skala des gewählten Features.
            \item \texttt{<<implements>>getFeatures() : Feature[]}
            \\ Liefert das ausgewählte Feature in einem Array.
        \end{itemize}
        \textbf{Attribute}
        \begin{itemize}
            \item \texttt{radius : int}
            \\ Radius, innerhalb dessen die Stationen angezeigt und gemittelt werden.
        \end{itemize}
    \end{Class}

    \begin{Class}{MapController}
        \textbf{Methoden}
        \begin{itemize}
            \item \texttt{handlePopup(pin : MapPin) : (ObservationStation, Observation)}
            \\ Wird aufgerufen wenn das Popup eines Stations-Markers geöffnet wird.
            \\ Liefert die zugehörige ObservationStation und die letzte Observation des aktuell gewählten Features zurück.
            \\ Existiert diese nicht wird (null, null) zurückgegeben.
            \item \texttt{handleViewportChange(viewport : Viewport)}
            \\ Wird aufgerufen wenn sich der Viewport der Leaflet-Karte ändert.
            \\ Aktualisiert gegebenenfalls private Eigenschaften.
            \bigskip
            \item \texttt{getPins() : MapPin[]}
            \\ Liefert die MapPins entsprechend der aktuellen Konfiguration und des aktuellen \emph{viewport} zurück.
            \item \texttt{getPolygons() : Polygon[]}
            \\ Liefert die Polygone entsprechend der aktuellen Konfiguration und des aktuellen \emph{viewport} zurück.
            \bigskip
            \item \texttt{changeFeature(feature : Feature) : void}
            \\ Ändert das aktuell gewählte Feature und aktualisiert die Karte entsprechend.
            \item \texttt{onConfigurationChange(mapConf : MapConfiguration) : void}
            \\ Wird aufgerufen wenn sich die aktuelle Konfiguration ändert. Ersetzt die aktuelle Konfiguration durch \emph{mapConf}und aktualisiert die Karte.
            \item \texttt{search(searchTerm : string) : void}
            \\ Versucht zum \emph{searchTerm} eine entsprechende Position zu finden und setzt die Mitte des Viewports darauf.
            Schlägt die Suche fehl wird ein Warnung ausgegeben.
            \item \texttt{updateCurrentPosition(position : Position) : void}
            \\ Setzt die Mitte des Viewports auf \emph{Position} und aktualisiert die Karte. 
        \end{itemize}
    \end{Class}

