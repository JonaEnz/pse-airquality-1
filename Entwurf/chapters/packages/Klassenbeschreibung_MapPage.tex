\subsection{MapPage}

    \begin{Class}{Viewport}
        \textbf{Methoden}
        \begin{itemize}
            \item \texttt{getCenter() : Position}
            \\ Gibt die aktuelle Position des Zentrums zurück.
            \item \texttt{setCenter(position : Position) : void}
            \\ Setzt \emph{position} als das Zentrum fest.
            \item \texttt{getZoom() : float}
            \\ Liefert den Zoom des Viewports.
            \item \texttt{setZoom(zoom : float) : void}
            \\ Setzt den Zoom des Viewports auf \emph{zoom}.
            \item \texttt{getRadius() : int}
            \\ Liefert den kleinsten Radius des Kreises in Koordinateneinheiten um das Zentrum,
            der den kompletten Anzeigebereich des Viewport einschließt.
        \end{itemize}
    \end{Class}

    \begin{Class}{Position}
        \textbf{Methoden}
        \begin{itemize}
            \\ Die gleiche Klasse gibt es schon in Model. Erfüllt sie einen unterschiedlichen Zweck? Ansonsten löschen!!1!
            \item \texttt{getCoordinates() : (lng : float, lat : float)}
            \\ Liefert die Koordinaten der Position als Tupel.
            \item \texttt{setCoordinates(lng : float, lat : float)}
            \\ Setzt die Koordinaten auf lng und lat fest.
            \\ Ist einer der beiden Parameter \emph{null} bleibt dieser unverändert, der andere wird angepasst.
            \item \texttt{toString() : string}
            \\ Liefert die Position als string im Dezimalgrad-Format.
        \end{itemize}
    \end{Class}

    \begin{Class}{<<abstract>> MapConfiguration}
        \textbf{Methoden}
        \begin{itemize}
            \item \texttt{getPins(port : Viewport) : MapPin[]}
            \\ Gibt MapPins innerhalb (oder nahe) des Viewports zurück.
            \item \texttt{getPolygons(port: Viewport) : Polygon[]}
            \\ Gibt Polygone innerhalb (oder nahe) des Viewports zurück.
            \item \texttt{getScale() : Scale}
            \\ Gibt die Skala der Konfiguration zurück.
            \item \texttt{getFeatures() : Feature[]}
            \\ Gibt die durch die Konfiguration spezifizierten Features zurück.
            \\ \emph{[]}, wenn keine Features dargestellt werden
        \end{itemize}
    \end{Class}

    \begin{Class}{StationConfiguration extends MapConfiguration}
        Eine Kartenkonfiguration die die Stationen mit Farbe nach einer Skala anzeigt.
        \textbf{Methoden}
        \begin{itemize}
            \item \texttt{setFeature(features : Feature) : void}
            \\ Legt das Feature der Konfiguration fest.
            \item \texttt{<<implements>>getPins(port : Viewport) : MapPin[]}
            \\ Liefert MapPins innerhalb (oder nahe) des Viewports.
            \\ Es werden nur Stationen angezeigt die das ausgewählte Feature besitzen
            \item \texttt{<<implements>>getPolygons(port: Viewport) : Polygon[]}
            \\ Liefert [].
            \item \texttt{<<implements>>getScale() : Scale}
            \\ Liefert die Skala des gewählten Features.
            \item \texttt{<<implements>>getFeatures() : Feature[]}
            \\ Liefert das ausgewählte Feature in einem Array.
        \end{itemize}
    \end{Class}

    \begin{Class}{PolygonConfiguration extends MapConfiguration}
        Eine Kartenkonfiguration die zwischen den Stationen Polygone aufspannt und sie nach den gemittelten Werten einfärbt.
        \textbf{Methoden}
        \begin{itemize}
            \item \texttt{setFeature(feature : Feature) : void}
            \\ Legt das Feature der Konfiguration fest.
            \item \texttt{<<implements>>getPins(port : Viewport) : MapPin[]}
            \\ Liefert MapPins innerhalb (oder nahe) des Viewports.
            \\ Es werden nur Stationen angezeigt die das ausgewählte Feature besitzen
            \item \texttt{<<implements>>getPolygons(port: Viewport) : Polygon[]}
            \\ Liefert Polygone die zwischen Stationen im Viewport \emph{port} aufgespannt sind.
            \item \texttt{<<implements>>getScale() : Scale}
            \\ Liefert die Skala des gewählten Features.
            \item \texttt{<<implements>>getFeatures() : Feature[]}
            \\ Liefert das ausgewählte Feature in einem Array.
        \end{itemize}
    \end{Class}

    \begin{Class}{NearConfiguration extends MapConfiguration}
        Eine Kartenkonfiguration die die Stationen mit Farbe nach einer Skala anzeigt.
        \textbf{Methoden}
        \begin{itemize}
            \item \texttt{setFeature(feature : Feature) : void}
            \\ Legt das Feature der Konfiguration fest.
            \item \texttt{<<implements>>getPins(port : Viewport) : MapPin[]}
            \\ Liefert MapPins innerhalb des Radius \emph{radius} um die Mitte des Viewports.
            \\ Es werden nur Stationen angezeigt die das ausgewählte Feature besitzen.
            \\ Die Färbung wird anhand des Mittelwerts der Stationen im Umkreis von \emph{radius} km festgelegt.
            \\ Dabei gibt der höchste Wert die Farbe am oberen Ende der Skala an, der niedrigste das untere Ende.
            \item \texttt{<<implements>>getPolygons(port: Viewport) : Polygon[]}
            \\ Liefert immer [].
            \item \texttt{<<implements>>getScale() : Scale}
            \\ Liefert die Skala des gewählten Features.
            \item \texttt{<<implements>>getFeatures() : Feature[]}
            \\ Liefert das ausgewählte Feature in einem Array.
        \end{itemize}
        \textbf{Attribute}
        \begin{itemize}
            \item \texttt{radius : int}
            \\ Radius, innerhalb dessen die Stationen angezeigt und gemittelt werden.
        \end{itemize}
    \end{Class}

    \begin{Class}{MapPin}
        \textbf{Methoden}
        \begin{itemize}
            \item \texttt{getPosition() : Position}
            \\ Liefert die Position des Pins
            \item \texttt{setPosition(position : Position) : void}
            \\ Setzt die Position des Pins

            \bigskip
            \item \texttt{getValue() : float}
            \\ Todo: Genaue Dokumentation für value.
            \\ Liefert den Wert anhand dem die Färbung bestimmt wird.
            \item \texttt{setValue(value : float) : void}
            \\ Setzt den Wert des Pins.

            \bigskip
            \item \texttt{getStationId() : string}
            \\ Liefert die interne Id der Station, die durch den Pin repräsentiert wird.
            \item \texttt{setStationId(id : string) : void}
            \\ Setzt die interne Id der Station, die durch den Pin repräsentiert wird.

            \bigskip
            \item \texttt{getColor() : Color}
            \\ Liefert die Farbe des Pins.
            \item \texttt{setColor(color : Color) : void}
            \\ Legt die Farbe des Pins fest.
        \end{itemize}
    \end{Class}

    \begin{Class}{Polygon}
        \textbf{Attribute}
        \begin{itemize}
            \item \texttt{stations : ObservationStation[]}
            \\ Stationen, die die Ecken des Polygons bilden.
            \item \texttt{color : Color}
            \\ Farbe des Polygons.
        \end{itemize}
    \end{Class}

