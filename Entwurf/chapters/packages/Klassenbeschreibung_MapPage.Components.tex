\subsection{MapPage.Components}

    \begin{Class}{MapView extends React.Component}
        \textbf{Methoden}
        \begin{itemize}
            \item \texttt{selectStation(station : ObservationStation)}
            \\ Legt die angegebene ObservationStation als ausgewählt fest.
            \item \texttt{getValueAt(position: Position, feature : Feature) : float}
            \\ Berechnet mithilfe naher Messstationen einen Schätzwert des gewählten Features an einer bestimmter Position.
            \\ Rückgabe:
            \\ \emph{null}: Es konnte kein Wert bestimmt werden da der Punkt zu weit entfernt von Messstationen ist.
            \\ \emph{float}: Der geschätzte Wert
            \item \texttt{openPopUp(station : ObservationStation) : void}
            \\ Öffnet die StationInfo in Map mit den Daten aus \emph{observation}.
            \\ Tausche die Daten aus falls \emph{StationInfo} bereits geöffnet ist.
            \item \texttt{closePopUp() : void}
            \\ Schließt die StationInfo in Map falls sie geöffnet ist.
        \end{itemize}
        \textbf{Attribute}
        \begin{itemize}
            \item \texttt{loading : boolean}
            \\ \emph{true}: Über der Map wird ein Ladesymbol angezeigt.
            \\ \emph{false}: Alle Kinds-Komponenten von MapView haben aktuelle Anzeigedaten.
            \\ \emph{null}: Nicht definiert.
            \item \texttt{selectedFeature : Feature}
            \\ \emph{null}: Die aktuelle \emph{MapConfiguration} unterstützt keine Features als Parameter.
            \\ \emph{Feature}: Die MapConfiguration zeigt aktuell Daten für das gewählte Feature an.
        \end{itemize}
    \end{Class}

    \begin{Class}{FeatureSelect extends React.Component}
        \textbf{Methoden}
        \begin{itemize}
            \item \texttt{getSelectedFeature() : Feature}
            \\ Rückgabe:
            \\ \emph{null}: Fehler, nicht definiert
            \\ \emph{Feature}: aktuell ausgewähltes Feature
            \item \texttt{setFeature(feature : Feature) : void}
            \\ \emph{feature}: Unterstütztes Feature der aktuellen Konfiguration
            \\ Legt den aktuell ausgewählten Eintrag fest.
        \end{itemize}
    \end{Class}

    \begin{Class}{Search extends React.Component}
        \textbf{Methoden}
        \begin{itemize}
            \item \texttt{find() : Position}
            \\ Rückgabe:
            \\ \emph{null}: Der aktuelle \emph{searchTerm} ist keine gültige Adresse und konnte nicht in eine umgewandelt werden.
            \\ \emph{Position}: Die Position des \emph{searchTerm} als Koordinaten.
            \item \texttt{findCurrentPosition() : Position}
            \\ Rückgabe:
            \\ Falls Positionsanfragen vom Nutzer erlaubt wurden: Die aktuelle Position des Nutzers
            \\ Falls Positionsanfragen vom Nutzer NICHT erlaubt wurden: null
            \item \texttt{reset() : void}
            \\ Setzt searchTerm auf das leere Wort.
        \end{itemize}
        \textbf{Attribute}
        \begin{itemize}
            \item \texttt{searchTerm : string}
            \\ Der aktuelle Inhalt der Textbox der Komponente.
        \end{itemize}
    \end{Class}
    
    \begin{Class}{Map extends React.Component}
        \textbf{Methoden}
        \begin{itemize}
            \item \texttt{setPosition(position : Position) : void}
            \\ Falls \emph{position} \emph{null}: Fehler
            \\ Sonst wird die Leaflet-Karte auf die angegebene Position zentriert.
            \item \texttt{setZoom(zoom : int) : void}
            \\ Setzt den Zoom der Leaflet-Karte auf \emph{zoom}.
        \end{itemize}
        \textbf{Attribute}
        \begin{itemize}
            \item \texttt{viewport : Viewport}
            \\ Der aktuelle Viewport der Leaflet-Karte
        \end{itemize}
    \end{Class}
    
    \begin{Class}{Legend extends React.Component}
        \textbf{Attribute}
        \begin{itemize}
            \item \texttt{scale : Skala}
            \\ Die aktuell verwendete Skala.
        \end{itemize}
    \end{Class}
    
    \begin{Class}{StationInfo extends React.Component}
        \textbf{Attribute}
        \begin{itemize}
            \item \texttt{openDetails : MaterialUI.Button}
            \\ click() öffnet die Detailansicht mit der ID der \emph{station}
            \item \texttt{station : Messstation}
            \\ Die Messstation zu der die StationInfo gehört.
            \item \texttt{isOpen : boolean}
            \\ \emph{true}: StationInfo wird bei der Position der \emph{station} angezeigt
            \\ \emph{false}: StationInfo ist ausgeblendet
        \end{itemize}
    \end{Class}

