\subsection{FROST}

\begin{Class}{RequestClass}
    \textbf{Methoden:}
    \begin{itemize}
        \item \texttt{getLatestObservations(center: Position, radius: int, feature: Feature): Observation[]}
        \\Gibt alle Observations im gegebenen Bereich zurück.
        \item \texttt{getStationsIn(middle: Position, radius: int): Station[]}
        \\Gibt alle Messstationen im gegebenen Bereich zurück.
        \item \texttt{getHistoricalObservations(station: Station, start: Date, end: Date, feature: Feature): Observation[]}
        \\Gibt alle Observations einer Station in einem gegebenen Zeitraum für das gegebene Feature zurück.
        \item \texttt{getHistoricalObservations(station: Station, start: Date, end: Date, frequency: Frequency, feature: Feature): Observation[]}
        
    \end{itemize}
\end{Class}

\begin{Class}{FROSTServer}
    \textbf{Methoden}
    \begin{itemize}
        \item \texttt{setUrl(url : string) : void}
        \\Setzt die URL des FROSTServer ein.
        \item \texttt{getUrl() : string}
        \\Gibt die URL des FROSTServer zurück.
    \end{itemize}
    
    \textbf{Attribute}
    \begin{itemize}
        \item //Todo
    \end{itemize}
\end{Class}

\begin{Class}{FROSTQuery}
    \textbf{Methoden}
    \begin{itemize}
        \item \texttt{send() : QueryResult}
        \\Schickt die Query an den Server. Gibt die Antwort als QueryResult zurück.
        \item \texttt{setTop(n : int) : void}
        \\Stellt die maximale Anzahl von Objekten ein.
        \item \texttt{setSkip(n : int) : void}
        \\Stellt die Anzahl der zu überspringenden Objekten ein.
        \item \texttt{setCount(set : bool) : void}
        \\Stellt ein ob die Anzahl der Objekte zurück gegeben werden soll.
        \item \texttt{setOrderBy(orderBy : string) : void}
        \\TODO
        \item \texttt{setSelect(select : string) : void}
        \\Spezifiziert welche Attribute in der Antwort enthalten sein sollen.
        \item \texttt{setFilter(filter : string) : void}
        \\TODO
        \item \texttt{setExpand(expand : string) : void}
        \\Stellt ein ein verschachtelten Objekttypen noch zurück gegeben werden sollen.
        \item \texttt{setID(id : string) : void}
        \\TODO
    \end{itemize}
    Für die Standardwerte der set-Methoden siehe:
    \\https://fraunhoferiosb.github.io/FROST-Server/sensorthingsapi/STA-Tailoring-Responses.html
\end{Class}

\begin{Class}{ObservationQuery}
    \textbf{Methoden}
    \begin{itemize}
        \item //Todo
    \end{itemize}
    
    \textbf{Attribute}
    \begin{itemize}
        \item //Todo
    \end{itemize}
\end{Class}

\begin{Class}{DatastreamQuery}
    \textbf{Methoden}
    \begin{itemize}
        \item //Todo
    \end{itemize}
    
    \textbf{Attribute}
    \begin{itemize}
        \item //Todo
    \end{itemize}
\end{Class}

\begin{Class}{ObservedPropertyQuery}
    \textbf{Methoden}
    \begin{itemize}
        \item //Todo
    \end{itemize}
    
    \textbf{Attribute}
    \begin{itemize}
        \item //Todo
    \end{itemize}
\end{Class}

\begin{Class}{ThingQuery}
    \textbf{Methoden}
    \begin{itemize}
        \item //Todo
    \end{itemize}
    
    \textbf{Attribute}
    \begin{itemize}
        \item //Todo
    \end{itemize}
\end{Class}

\begin{Class}{LocationQuery}
    \textbf{Methoden}
    \begin{itemize}
        \item //Todo
    \end{itemize}
    
    \textbf{Attribute}
    \begin{itemize}
        \item //Todo
    \end{itemize}
\end{Class}

\begin{Class}{ObservationResult}
    \textbf{Methoden}
    \begin{itemize}
        \item //Todo
    \end{itemize}
    
    \textbf{Attribute}
    \begin{itemize}
        \item //Todo
    \end{itemize}
\end{Class}

\begin{Class}{DatastreamResult}
    \textbf{Methoden}
    \begin{itemize}
        \item //Todo
    \end{itemize}
    
    \textbf{Attribute}
    \begin{itemize}
        \item //Todo
    \end{itemize}
\end{Class}

\begin{Class}{ObservedPropertyResult}
    \textbf{Methoden}
    \begin{itemize}
        \item //Todo
    \end{itemize}
    
    \textbf{Attribute}
    \begin{itemize}
        \item //Todo
    \end{itemize}
\end{Class}

\begin{Class}{ThingResult}
    \textbf{Methoden}
    \begin{itemize}
        \item //Todo
    \end{itemize}
    
    \textbf{Attribute}
    \begin{itemize}
        \item //Todo
    \end{itemize}
\end{Class}

\begin{Class}{All}
    \textbf{Methoden}
    \begin{itemize}
        \item //Todo
    \end{itemize}
    
    \textbf{Attribute}
    \begin{itemize}
        \item //Todo
    \end{itemize}
\end{Class}

\begin{Class}{DataProvider}
    \textbf{Methoden}
    \begin{itemize}
        \item \texttt{getLatestObservations(center : Position, radius: int, feature : Feature) : Observation[]}
        \item \texttt{getStationsIn(middle: Position, radius : int) : Station[]}
        \item \texttt{getHistoricalObservations(station : Station, start : Date, end : Date, feature : Feature) : Observation[]}
        \item \texttt{getHistoricalObservations(station : Station, start : Date, end : Date, frequency : Frequency, feature : Feature) : Observation[]}
        \item \texttt{getScale(feature : Feature) : Scale}
        \item \texttt{getStation(id : string) : Station}
    \end{itemize}
    
    \textbf{Attribute}
    \begin{itemize}
        \item //Todo
    \end{itemize}
\end{Class}