\subsection{DetailPage}

   
    \begin{Class}{ObservationStationInfo}
      
            \\ Information über die Station für die Nutzer.
    \end{Class}

    \begin{Class}{StaticMap}
            \\Eine statische Karte soll angezeigt werden, die die Position des Sensors anzeigt.
    \end{Class}

  
    \begin{Class}{FeatureHistoryLineChart}
           \\ Graph zeigt die Veränderung der Messdaten eines Feature
    \end{Class}

    \begin{Class}{ComparisonLineChart}
            \\Vergleicht einen Zeitraum mit dem vorherigen Zeitraum.
    \end{Class}

    \begin{Class}{ComparisonToLastYearPieChart}
            \\ PieChart, die die Grenzwertüberschreitungen im aktuellen Jahr mit denen aus dem letzten jahr vergleict 
    \end{Class}

    \begin{Class}{Diagram}
        \textbf{Methoden}
        \begin{itemize}
            \item \texttt{draw() : void}
            \item Übergibt die Daten an Google Charts, womet der das Diagramm gezeichnet wird.
            \item \texttt{setTimespan(timespan : Timespan) : void} 
            \item ??? 
            \item \texttt{setObservationStation(observationStation : Model.ObservationStation) : void}
            \item \texttt{loadData() : void}
        \end{itemize}
        
        \textbf{Attribute}
        \begin{itemize}
            \item \texttt{fature : Feature}
            \item Das Diagramm modelliert immer nur ein Feature.
        \end{itemize}
    \end{Class}

