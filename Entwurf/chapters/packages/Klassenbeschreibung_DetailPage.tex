\subsection{DetailPage}
\\Das ist das Detailansicht-Component. Die Detailansicht von der Kartenansicht aus geöffnet, wenn der Nutzer auf eine Messtation klickt. Hier werden dann Diagramme und eine statische Karte angezeigt.
   
    \begin{Class}{ObservationStationInfo}
      
            \\ Information über die Station für die Nutzer.
    \end{Class}

    \begin{Class}{StaticMap}
            \\Eine statische Karte soll angezeigt werden, die die Position des Sensors anzeigt.
    \end{Class}

  
    \begin{Class}{FeatureHistoryLineChart}
           \\ Graph zeigt die Veränderung der Messdaten eines Feature
    \end{Class}

    \begin{Class}{ComparisonLineChart}
            \\Vergleicht einen Zeitraum mit dem vorherigen Zeitraum.
    \end{Class}

    \begin{Class}{ComparisonToLastYearPieChart}
            \\ PieChart, die die Grenzwertüberschreitungen im aktuellen Jahr mit denen aus dem letzten jahr vergleict 
    \end{Class}

    \begin{Class}{Diagram}
        \\Die Daten werden in der Klasse Diagram an Google Charts weitergereicht und dort wird ein Graph/Kuchendiagramm erstellt.
        \textbf{Attribute}
        \begin{itemize}
            \item \texttt{props.feature : Feature}
            \\Das Diagramm modelliert immer genau ein Feature.
            \item \texttt{props.observationStation : ObservationStation}
            \\ Das Diagram modelliert den Vrlauf für genau eine Messstaion.

        \end{itemize}
    \end{Class}

