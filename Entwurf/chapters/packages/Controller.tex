\subsection{Controller}

\subsubsection{FROST}

\begin{Class}{FROSTQuery}
    \textbf{Methoden}
    \begin{itemize}
        \item \texttt{send() : }
        \\ Todo: Ist nicht asyncron(?), was wird zurückgegeben
        \\Sendet die Abfrage an den Server und gibt die Antwort als QueryResult zurück.
        \item \texttt{setTop(n : int) : void}
        \\Stellt die maximale Anzahl von Objekten ein.
        \item \texttt{setSkip(n : int) : void}
        \\Stellt die Anzahl der zu überspringenden Objekte ein.
        \item \texttt{enableCount(set : boolean) : void}
        \\Stellt ein, ob die Anzahl der Objekte zurückgegeben werden soll.
        \item \texttt{setOrderBy(orderBy : string) : void}
        \\TODO
        \item \texttt{setSelect(select : string) : void}
	\\Todo: Hat select ein spezielles Format? Soll das nicht lieber ein Array sein?
        \\Spezifiziert, welche Attribute in der Antwort enthalten sein sollen.
        \item \texttt{setFilter(filter : string) : void}
        \\TODO
        \item \texttt{setExpand(expand : string) : void}
        \\Stellt ein, ob verschachtelte Objekttypen zurückgegeben werden sollen.
        \item \texttt{setId(id : string) : void}
        \\TODO
    \end{itemize}
    Für die Standardwerte der set-Methoden siehe: \url{https://fraunhoferiosb.github.io/FROST-Server/sensorthingsapi/STA-Tailoring-Responses.html}
\end{Class}

\begin{Class}{FROSTServer}
    \textbf{Methoden}
    \begin{itemize}
        \item \texttt{setUrl(url : string) : void}
        \\Setzt die URL des FROSTServers ein.
        \item \texttt{getUrl() : string}
        \\Gibt die URL des FROSTServers zurück. njskdngkj
    \end{itemize}
    
    \textbf{Attribute}
    \begin{itemize}
        \item \texttt{url : string}
        \\Die URL des FROSTServers.
    \end{itemize}
\end{Class}

\begin{Class}{DataProvider}
    \textbf{Methoden}
    \begin{itemize}
        \item \texttt{getLatestObservations(center : Position, radius: int, feature : Feature) : Observation[]}
        \\Gibt alle Observations im spezifizierten Bereich zurück
        \item \texttt{getStationsIn(middle: Position, radius : int) : Station[]}
        \\TODO: brauchen wir diese Methode?
        \item \texttt{getHistoricalObservations(station : Station, start : Date, end : Date, feature : Feature) : Observation[]}
        \\Gibt Observations der gegebenen Station, zwischen start und end Zeitpunkt, des gegebenen Features zurück.
        \item \texttt{getHistoricalObservationsFreq(station : Station, start : Date, end : Date, frequency : Frequency, feature : Feature) : Observation[]}
        \\Gibt Observations der gegebenen Station, zwischen start und end Zeitpunkt mit gegebener Frequenz, des gegebenen Features zurück.
        \item \texttt{getScale(feature : Feature) : Scale}
        \\Gibt die Skala des gegebenen Features zurück.
        \item \texttt{getStation(id : string) : Station}
        \\Gibt die Station die zur gegebenen ID gehört zurück.
    \end{itemize}
    
    \textbf{Attribute}
    \begin{itemize}
        \item //Todo
    \end{itemize}
\end{Class}
