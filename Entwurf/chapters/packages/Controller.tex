\subsection{Controller}

\subsubsection{FROST}

\begin{Class}{FROSTQuery}
    \textbf{Methoden}
    \begin{itemize}
        \item \texttt{send() : }
        \\ Todo: Ist nicht asyncron(?), was wird zurückgegeben
        \\Sendet die Abfrage an den Server und gibt die Antwort als QueryResult zurück.
        \item \texttt{setTop(n : int) : void}
        \\Stellt die maximale Anzahl von Objekten ein.
        \item \texttt{setSkip(n : int) : void}
        \\Stellt die Anzahl der zu überspringenden Objekte ein.
        \item \texttt{enableCount(set : boolean) : void}
        \\Stellt ein, ob die Anzahl der Objekte zurückgegeben werden soll.
        \item \texttt{setOrderBy(orderBy : string) : void}
        \\TODO
        \item \texttt{setSelect(select : string) : void}
	\\Todo: Hat select ein spezielles Format? Soll das nicht lieber ein Array sein?
        \\Spezifiziert, welche Attribute in der Antwort enthalten sein sollen.
        \item \texttt{setFilter(filter : string) : void}
        \\TODO
        \item \texttt{setExpand(expand : string) : void}
        \\Stellt ein, ob verschachtelte Objekttypen zurückgegeben werden sollen.
        \item \texttt{setId(id : string) : void}
        \\TODO
    \end{itemize}
    Für die Standardwerte der set-Methoden siehe: \url{https://fraunhoferiosb.github.io/FROST-Server/sensorthingsapi/STA-Tailoring-Responses.html}
\end{Class}

\begin{Class}{FROSTServer}
    \textbf{Methoden}
    \begin{itemize}
        \item \texttt{setUrl(url : string) : void}
        \\Setzt die URL des FROSTServers ein.
        \item \texttt{getUrl() : string}
        \\Gibt die URL des FROSTServers zurück. njskdngkj
    \end{itemize}
    
    \textbf{Attribute}
    \begin{itemize}
        \item \texttt{url : string}
        \\Die URL des FROSTServers.
    \end{itemize}
\end{Class}

\begin{Class}{DataProvider}
    \textbf{Methoden}
    \begin{itemize}
        \item \texttt{getLatestObservations(center : Position,
        \\radius: int, feature : Feature) : Observation[]}
        \\Gibt alle Observations im spezifizierten Bereich zurück
        \item \texttt{getStationsIn(middle: Position,
        \\radius : int) : Station[]}
        \\TODO: brauchen wir diese Methode?
        \item \texttt{getHistoricalObservations(
            \\station : Station, start : Date, end : Date,
            \\feature : Feature) : Observation[]}
        \\Gibt Observations der gegebenen Station, zwischen start und end Zeitpunkt, des gegebenen Features zurück.
        \item \texttt{getHistoricalObservationsFreq(
            \\station : Station, start : Date, end : Date,
            \\frequency : Frequency, feature : Feature
            \\) : Observation[]}
        \\Gibt Observations der gegebenen Station, zwischen start und end Zeitpunkt mit gegebener Frequenz, des gegebenen Features zurück.
        \item \texttt{getScale(feature : Feature) : Scale}
        \\Gibt die Skala des gegebenen Features zurück.
        \item \texttt{getStation(id : string) : Station}
        \\Gibt die Station die zur gegebenen ID gehört zurück.
    \end{itemize}
    
    \textbf{Attribute}
    \begin{itemize}
        \item //Todo
    \end{itemize}
\end{Class}

\subsubsection{MapPage}
\begin{Class}{MapController}
    \textbf{Methoden}
    \begin{itemize}
        \item \texttt{handlePopup(pin : MapPin) : [Station, Observation]}
        \\ Holt sich die zum \emph{pin} gehörige ObservationStation vom FROST-Server,
        holt den letzten Messwert des aktuell gewählten Features und gibt beides zurück.
        \item \texttt{handleViewportChange(viewport : Viewport)}
        \\ Aktualisiert den aktuellen Viewport auf \emph{viewport}.

        \bigskip
        \item \texttt{getPins() : MapPin[]}
        \\ Holt die Pins der aktuellen MapConfiguration mit dem aktuellen Viewport 
        und gibt sie zurück.
        \item \texttt{getPolygons() : Polygon[]}
        \\ Holt die Polygone der aktuellen MapConfiguration mit dem aktuellen Viewport 
        und gibt sie zurück.

        \bigskip
        \item \texttt{changeFeature(feature : Feature) : void}
        \\ Wechselt das aktuell gewählte Feature auf \emph{feature}.
        \item \texttt{onConfigurationChange(mapConf : MapConfiguration) : void}
        \\ Behandelt eine Änderung der Kartenkonfiguration.
        Die aktuelle Konfiguration wird auf \emph{mapConf} gesetzt und die Pins und Polygone der Karte aktualisiert.
        \item \texttt{search(searchTerm : string) : void}
        \\ Versucht den \emph{searchTerm} als Position zu interpretieren und setzt die Mitte des Viewports darauf.
        Wenn die Interpretation nicht möglich ist wird eine Warnung ausgegeben.
        \item \texttt{updateCurrentPosition(position : Position) : void}
        \\ Setzt die aktuelle Position auf \emph{position} und aktualisiert die Karte.
    \end{itemize}
\end{Class}

\begin{Class}{MapConfiguration}
    \textbf{Methoden}
    \begin{itemize}
        \item \texttt{getPins(port : Viewport) : MapPin[]}
        \\ Holt Messstationen die eines der gewählten Features unterstützen und im \emph{viewport} liegen vom Server,
        erstellt daraus MapPins und gibt sie zurück.
        \item \texttt{getPolygons(port: Viewport) : Polygon[]}
        \\ Wie getPins(v), nur mit Polygonen. 
        \item \texttt{getScale() : Scale}
        \\ Gibt die aktuell aktive Skala zurück.
        \item \texttt{getFeatures() : Feature[]}
        \\ Die aktuell gewählten Features die von der Konfiguration benötigt werden.
    \end{itemize}
\end{Class}

\begin{Class}{StationConfiguration extends MapConfiguration}
    Eine Konfiguration bei der die einzelnen Messstationen als Marker dargestellt werden.
    \textbf{Methoden}
    \begin{itemize}
        \item \texttt{getPins(port : Viewport) : MapPin[]}
        \\ Messstationen, als Pins repräsentiert.
        erstellt daraus MapPins und gibt sie zurück.
        \item \texttt{getPolygons(port: Viewport) : Polygon[]}
        \\ Gibt [] zurück.
        \bigskip
        \item \texttt{setFeature(feature : Feature)}
        \\ Setzt die gewählten Features auf [\emph{feature}].
    \end{itemize}
\end{Class}

\begin{Class}{PolygonConfiguration extends MapConfiguration}
    \textbf{Methoden}
    \begin{itemize}
        \item \texttt{getPins(port : Viewport) : MapPin[]}
        \\ Gibt [] zurück.
        erstellt daraus MapPins und gibt sie zurück.
        \item \texttt{getPolygons(port: Viewport) : Polygon[]}
        \\ Spannt zwischen jeweils drei Stationen Polygone auf.
        Die Polygone werden in der Farbe gefärbt die entsprechend der Skala für den Mittelwert der einzelnen letzten Messwerte entsprechen.
        \bigskip
        \item \texttt{setFeature(feature : Feature)}
        \\ Setzt die gewählten Features auf [\emph{feature}].
    \end{itemize}
\end{Class}

\begin{Class}{NearConfiguration extends MapConfiguration}
    \textbf{Methoden}
    \begin{itemize}
        \item \texttt{getPins(port : Viewport) : MapPin[]}
        \\ Die Messstationen werden als Pins repräsentiert die entsprechend der Skala gefärbt sind.
        \\ Dabei wird allerdings der Messwert so skaliert dass der niedrigste Messwert im Radius 0 ist, der höchste die größte Zahl der Skala.
        erstellt daraus MapPins und gibt sie zurück.
        \item \texttt{getPolygons(port: Viewport) : Polygon[]}
        \\ Gibt [] zurück.
        \bigskip
        \item \texttt{setFeature(feature : Feature)}
        \\ Setzt die gewählten Features auf [\emph{feature}].
    \end{itemize}
    \textbf{Attribute}
    \begin{itemize}
        \item \texttt{Radius}
        \\ Der Radius aus dem die Daten für die Skalierung geholt werden.
    \end{itemize}
\end{Class}