\subsection{Model}

    \begin{Class}{ObservationStation}
        \textbf{Methoden}
        \begin{itemize}
            \item \texttt{Todo}
            \item \texttt{getName() : string}
            \\ Der Name der Station.
            \item \texttt{getFeatures() : List<Feature>}
            \\ Liste an Features die die Station erfassen kann und von der Anwendung unterstützt werden.
            \item \texttt{getPosition() : Position}
            \\ Die Position an der sich die Station befindet.
            \item \texttt{getAverage() : float}
            \\ Der Durchschnitt der Messwerte des letzten Tages.
            \item \texttt{hasFeature(feature : Feature) : boolean}
            \\ \emph{true} wenn die Station das Feature erfassen kann.
            \\ \emph{false} sonst.
        \end{itemize}
    \end{Class}

    \begin{Class}{Observation}
        
        \textbf{Methoden}
        \begin{itemize}
            \item \texttt{getFeature() : Feature}
            \\ Gibt das Feature der Beobachtung zurück.
            \item \texttt{getDate() : Date}
            \\ Gibt den Zeitpunkt der Beobachtung zurück.
            \item \texttt{getValue() : float}
            \\ Gibt den Wert der Beobachtung (in der Einheit des Feature) zurück.
        \end{itemize}
    \end{Class}

    \begin{Class}{Position}
        \textbf{Methoden}
        \begin{itemize}
            \item \texttt{getCoordinates() : (lng : float, lat : float)}
            \\ Liefert die Koordinaten im Dezimalgrad als Tupel zurück.
            \item \texttt{setCoordinates(lng : float, lat : float)}
            \\ Setzt die Koordinaten fest. Die Eingaben werden als Dezimalgrad interpretiert.

            \item \texttt{toString() : string}
            \\ Gibt die Daten im Format 'N 0.00000°, O 0.00000°' als Zeichenkette zurück.
        \end{itemize}
        
        \textbf{Attribute}
        \begin{itemize}
            \item \texttt{Todo}
        \end{itemize}
    \end{Class}

    \begin{Class}{Feature}
        \textbf{Methoden}
        \begin{itemize}
            \item \texttt{getId() : string}
            \\ Gibt die FROST-Id des Features zurück.
            \item \texttt{getNameId() : string}
            \\ Gibt die Id des Featurenamens für die Sprachdatei zurück.
            \item \texttt{getUnitOfMeasurement() : string}
            \\ Gibt die Einheit in der das Feature angegeben ist zurück.
            \item \texttt{getDescriptionId() : string}
            \\ Gibt die Id der Featurebeschreibung für die Sprachdatei zurück.
            \item \texttt{getDefaultScale() : Scale}
            \\ Gibt die Farbskala aus der Featurekonfiguration zurück.
            \item \texttt{getLimit() : float}
            \\ Das Limit für eine Warnmeldung. Wenn das Limit -1 ist wird nie eine Warnung ausgegeben.
            \item \texttt{isLimitExceeded(obs : Observation) : Boolean}
            \\ \emph{true} wenn der Wert der Observation das Limit des Features übersteigt.
            \\ \emph{false} wenn das Limit -1 ist oder der Wert der Observation <= Limit.
        \end{itemize}
    \end{Class}

    \begin{Class}{Color}
        \textbf{Methoden}
        \begin{itemize}
            \item \texttt{getRGB() : string}
            \\ Liefert den RGB-Wert der Farbe im Format '\#FFFFFF' zurück.
            \item \texttt{setRGB(hex : string) : void}
            \\ Speichert die Farbe mit RGB Wert \emph{hex}. Es wird nur das Format '\#FFFFFF' akzeptiert.
        \end{itemize}
    \end{Class}

    \begin{Class}{Scale}
        \textbf{Methoden}
        \begin{itemize}
            \item \texttt{getColor(value : float) : Color}
            \\ Gibt die Farbe des Wertes entsprechend der Konfiguration zurück.
            \item \texttt{addColor(color: Color, threshold : float)}
            \\ Fügt eine Farbe und eine entsprechende Grenze hinzu.
            \\ Gibt es bereits eine Farbe mit dieser Grenze wird sie überschrieben.
            \item \texttt{removeColor(threshold : float) : void}
            \\ Entfernt den Eintrag mit Grenze \emph{threshold}.
            \\ Gibt es einen solchen Eintrag nicht geschieht nichts.
            \item \texttt{getAllEntries() : (float, Color)[]}
            \\ Liefert ein Array an Tupeln aus Farben und Grenzwerten zurück.
            \\ Das Array ist aufsteigend nach Grenzwerten sortiert.
        \end{itemize}
    \end{Class}