\subsection{Model}

    \begin{Class}{ObservationStation}
        \textbf{Methoden}
        \begin{itemize}
            \item \texttt{getName() : string}
            \\ Der Name der Station.
            \item \texttt{getFeatures() : Feature[]}
            \\ Array von Features, die die Station erfassen kann, und von der Anwendung unterstützt werden.
            \item \texttt{getPosition() : Position}
            \\ Die Position an der sich die Station befindet.
            \item \texttt{getAverage() : number}
            \\ Der Durchschnitt der Messwerte des letzten Tages.
            \item \texttt{hasFeature(feature : Feature) : boolean}
            \\ \emph{true} wenn die Station das Feature erfassen kann.
            \\ \emph{false} sonst.
        \end{itemize}
    \end{Class}

    \begin{Class}{Observation}
        
        \textbf{Methoden}
        \begin{itemize}
            \item \texttt{getFeature() : Feature}
            \\ Gibt das Feature der Beobachtung zurück.
            \item \texttt{getDate() : Date}
            \\ Gibt den Zeitpunkt der Beobachtung zurück.
            \item \texttt{getValue() : number}
            \\ Gibt den Wert der Beobachtung (in der Einheit des Feature) zurück.
        \end{itemize}
    \end{Class}

    \begin{Class}{Position}
        \textbf{Methoden}
        \begin{itemize}
            \item \texttt{getCoordinates() : (lng : number, lat : number)}
            \\ Liefert die Koordinaten im Dezimalgrad als Tupel zurück.
            \item \texttt{setCoordinates(lng : number, lat : number)}
            \\ Setzt die Koordinaten fest. Die Eingaben werden als Dezimalgrad interpretiert.

            \item \texttt{toString() : string}
            \\ Gibt die Daten im Format 'N 0.00000°, O 0.00000°' als Zeichenkette zurück.
        \end{itemize}
    \end{Class}

    \begin{Class}{Feature}
        \textbf{Methoden}
        \begin{itemize}
            \item \texttt{getId() : string}
            \\ Gibt die FROST-Id des Features zurück.
            \item \texttt{getNameId() : string}
            \\ Gibt die Id des Featurenamens für die Sprachdatei zurück.
            \item \texttt{getUnitOfMeasurement() : string}
            \\ Gibt die Einheit in der das Feature angegeben ist zurück.
            \item \texttt{getDescriptionId() : string}
            \\ Gibt die Id der Featurebeschreibung für die Sprachdatei zurück.
            \item \texttt{getDefaultScale() : Scale}
            \\ Gibt die Farbskala aus der Featurekonfiguration zurück.
            \item \texttt{getLimit() : number}
            \\ Das Limit für eine Warnmeldung. Wenn das Limit -1 ist, wird nie eine Warnung ausgegeben.
            \item \texttt{isLimitExceeded(obs : Observation) : Boolean}
            \\ \emph{true} wenn der Wert der Observation das Limit des Features übersteigt.
            \\ \emph{false} wenn das Limit -1 ist oder der Wert der Observation <= Limit.
        \end{itemize}
    \end{Class}

    \begin{Class}{Position}
        \textbf{Methoden}
        \begin{itemize}
            \item \texttt{getCoordinates() : (lng : float, lat : float)}
            \\ Liefert die Koordinaten der Position als Tupel.
            \item \texttt{setCoordinates(lng : float, lat : float)}
            \\ Setzt die Koordinaten auf lng und lat fest.
            \\ Ist einer der beiden Parameter \emph{null} bleibt dieser unverändert, der andere wird angepasst.
            \item \texttt{toString() : string}
            \\ Liefert die Position als string im Dezimalgrad-Format.
        \end{itemize}
    \end{Class}

    \begin{Class}{Color}
        \textbf{Methoden}
        \begin{itemize}
            \item \texttt{toString() : string}
            \\ Liefert den RGB-Wert der Farbe im Format '\#FFFFFF' zurück.
            \item \texttt{setRGB(hex : string) : void}
            \\ Todo: Genauere Dokumentation erforderlich. Wohin wird die Farbe gespeichert? Soll das eine Art constructor sein? Braucht man überhaupt so eine Funktion, sollte die Farbe nicht lieber durch ein Zahlentripel kodiert werden?
            \\ Speichert die Farbe mit RGB Wert \emph{hex}. Es wird nur das Format '\#FFFFFF' akzeptiert.
        \end{itemize}
    \end{Class}

    \begin{Class}{Scale}
        \textbf{Methoden}
        \begin{itemize}
            \item \texttt{getColor(value : number) : Color}
            \\ Gibt die Farbe des Wertes entsprechend der Konfiguration zurück.
            \item \texttt{addColor(color: Color, threshold : number)}
            \\ Fügt eine Farbe und eine entsprechende Grenze hinzu.
            \\ Gibt es bereits eine Farbe mit dieser Grenze wird sie überschrieben.
            \item \texttt{removeColor(threshold : number) : void}
            \\ Todo: Genauere Dokumentation bezüglich des Grenzwerts. 
            \\ Entfernt den Eintrag mit dem Grenzwert \emph{threshold}.
            \\ Gibt es einen solchen Eintrag nicht, geschieht nichts.
            \item \texttt{getAllEntries() : (number, Color)[]}
            \\ Liefert ein Array an Tupeln von Grenzwerten und Farben zurück.
            \\ Das Array ist aufsteigend nach Grenzwerten sortiert.
        \end{itemize}
    \end{Class}

    \subsection{MapPage}

    \begin{Class}{Polygon}
        \textbf{Attribute}
        \begin{itemize}
            \item \texttt{stations : ObservationStation[]}
            \\ Stationen, die die Ecken des Polygons bilden.
            \item \texttt{color : Color}
            \\ Farbe des Polygons.
        \end{itemize}
    \end{Class}

    \begin{Class}{MapPin}
        \textbf{Methoden}
        \begin{itemize}
            \item \texttt{getPosition() : Position}
            \\ Liefert die Position des Pins
            \item \texttt{setPosition(position : Position) : void}
            \\ Setzt die Position des Pins

            \bigskip
            \item \texttt{getValue() : float}
            \\ Todo: Genaue Dokumentation für value.
            \\ Liefert den Wert anhand dem die Färbung bestimmt wird.
            \item \texttt{setValue(value : float) : void}
            \\ Setzt den Wert des Pins.

            \bigskip
            \item \texttt{getStationId() : string}
            \\ Liefert die interne Id der Station, die durch den Pin repräsentiert wird.
            \item \texttt{setStationId(id : string) : void}
            \\ Setzt die interne Id der Station, die durch den Pin repräsentiert wird.

            \bigskip
            \item \texttt{getColor() : Color}
            \\ Liefert die Farbe des Pins.
            \item \texttt{setColor(color : Color) : void}
            \\ Legt die Farbe des Pins fest.
        \end{itemize}
    \end{Class}
