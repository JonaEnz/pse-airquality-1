\section{Vorwort}
\\Dises Dokument modelliert die Architektur des Programms SmartAQNet. Wir als Entwickler-Team haben den Aufbau in Typescript components und Klassen aufgeteilt. Dabei war unser Ziel, das Programm möglichst stabil, flexibel und objektorientiert zu gestalten. Zusätzlich lässt sich die Architektur sehr gut in Model, View und Controller einteilen, indem wir die Querys, das Grundgerüst und die Benutzeroberfläche gut trennen können. Für die Erstellung der Diagramme wurde PlantUML und Visual Studio Code, für die Zusammenarbeit GitHub benutzt.
