\section{Datenmodell}
\subsection{Feature}

Unterstützte Features sind in JSON Dateien definiert.
Dies ermöglicht das einfache Hinzufügen von weiteren Features und die freie Konfiguration für die Betreiber der Seite.


\begin{lstlisting}[language=json,firstnumber=1]
    {
	    "id": string,
	    "nameId": string,
	    "unitOfMeasurement": string,
	    "descriptionId": string,
	    "limit": number,
	    "defaultScale": {
		    [number] : string,
	    },
	    "supportedDiagrams": string[]
    }
    
\end{lstlisting}

\begin{itemize}
	\item \texttt{id} 
	\\ Eine eindeutige Id, wird vom FROST-Server übernommen.
	\item \texttt{nameId} 
	\\ Der Schlüssel für den lokalisierten Namen des Features
	\item \texttt{unitOfMeasurement} 
	\\ Die Messeinheit des Features
	\item \texttt{descriptionId} 
	\\ Der Schlüssel für die lokalisierte Beschreibung
	\item \texttt{limit} 
	\\ Der Messwert ab dem eine Warnung ausgegeben wird
	\item \texttt{defaultScale} 
	\\ Eine Farbskala für das Feature, dient als Fallback falls keine Skala auf dem Server gefunden wird.
	\\ Bei \emph{[number] : string} ist der Schlüssel die untere Grenze für die Farbe.
	Die Farbe selbst ist in HEX codiert (bspw. '\#AAAAAA') 
\end{itemize}
\newpage
\subsection{Sprachdateien}


\begin{lstlisting}[language=json,firstnumber=1]
	{
		"id": string,
		"name": string,
		"strings": {
			[string]: string,
		}
	}
	
\end{lstlisting}

\begin{itemize}
	\item \texttt{id} 
	\\ Eine eindeutige Id für die Sprache.
	\item \texttt{name} 
	\\ Der Name der Sprache in der Sprache selbst.
	\item \texttt{strings} 
	\\ Eine Liste an Schlüsseln und der zugehörigen Zeichenkette in der Sprache.
\end{itemize}