\section{Klassenbeschreibung}

\begin{Class}{RequestClass}
    \textbf{Methoden:}
    \begin{itemize}
        \item \texttt{getLatestObservations(center: Position, radius: int, feature: Feature): Observation[]}
        \\Gibt alle Observations im gegebenen Bereich zurück.
        \item \texttt{getStationsIn(middle: Position, radius: int): Station[]}
        \\Gibt alle Messstationen im gegebenen Bereich zurück.
        \item \texttt{getHistoricalObservations(station: Station, start: Date, end: Date, feature: Feature): Observation[]}
        \\Gibt alle Observations einer Station in einem gegebenen Zeitraum für das gegebene Feature zurück.
        \item \texttt{getHistoricalObservations(station: Station, start: Date, end: Date, frequency: Frequency, feature: Feature): Observation[]}
        
    \end{itemize}
\end{Class}

\subsection{MapPage.Components}

\begin{Class}{MapView extends React.Component}
    \textbf{Methoden}
    \begin{itemize}
        \item \texttt{selectStation(station : ObservationStation)}
        \\ Legt die angegebene ObservationStation als ausgewählt fest.
        \item \texttt{getValueAt(position: Position, feature : Feature) : float}
        \\ Berechnet mithilfe naher Messstationen einen Schätzwert des gewählten Features an einer bestimmter Position.
        \\ Rückgabe:
        \\ \emph{null}: Es konnte kein Wert bestimmt werden da der Punkt zu weit entfernt von Messstationen ist.
        \\ \emph{float}: Der geschätzte Wert
        \item \texttt{openPopUp(station : ObservationStation) : void}
        \\ Öffnet die StationInfo in Map mit den Daten aus \emph{observation}.
        \\ Tausche die Daten aus falls \emph{StationInfo} bereits geöffnet ist.
        \item \texttt{closePopUp() : void}
        \\ Schließt die StationInfo in Map falls sie geöffnet ist.
    \end{itemize}
    \textbf{Attribute}
    \begin{itemize}
        \item \texttt{loading : boolean}
        \\ \emph{true}: Über der Map wird ein Ladesymbol angezeigt.
        \\ \emph{false}: Alle Kinds-Komponenten von MapView haben aktuelle Anzeigedaten.
        \\ \emph{null}: Nicht definiert.
        \item \texttt{selectedFeature : Feature}
        \\ \emph{null}: Die aktuelle \emph{MapConfiguration} unterstützt keine Features als Parameter.
        \\ \emph{Feature}: Die MapConfiguration zeigt aktuell Daten für das gewählte Feature an.
    \end{itemize}
\end{Class}

\begin{Class}{FeatureSelect extends React.Component}
    \textbf{Methoden}
    \begin{itemize}
        \item \texttt{getSelectedFeature() : Feature}
        \\ Rückgabe:
        \\ \emph{null}: Fehler, nicht definiert
        \\ \emph{Feature}: aktuell ausgewähltes Feature
        \item \texttt{setFeature(feature : Feature) : void}
        \\ \emph{feature}: Unterstütztes Feature der aktuellen Konfiguration
        \\ Legt den aktuell ausgewählten Eintrag fest.
    \end{itemize}
\end{Class}

\begin{Class}{Search extends React.Component}
    \textbf{Methoden}
    \begin{itemize}
        \item \texttt{find() : Position}
        \\ Rückgabe:
        \\ \emph{null}: Der aktuelle \emph{searchTerm} ist keine gültige Adresse und konnte nicht in eine umgewandelt werden.
        \\ \emph{Position}: Die Position des \emph{searchTerm} als Koordinaten.
        \item \texttt{findCurrentPosition() : Position}
        \\ Rückgabe:
        \\ Falls Positionsanfragen vom Nutzer erlaubt wurden: Die aktuelle Position des Nutzers
        \\ Falls Positionsanfragen vom Nutzer NICHT erlaubt wurden: null
        \item \texttt{reset() : void}
        \\ Setzt searchTerm auf das leere Wort.
    \end{itemize}
    \textbf{Attribute}
    \begin{itemize}
        \item \texttt{searchTerm : string}
        \\ Der aktuelle Inhalt der Textbox der Komponente.
    \end{itemize}
\end{Class}

\begin{Class}{Map extends React.Component}
    \textbf{Methoden}
    \begin{itemize}
        \item \texttt{setPosition(position : Position) : void}
        \\ Falls \emph{position} \emph{null}: Fehler
        \\ Sonst wird die Leaflet-Karte auf die angegebene Position zentriert.
        \item \texttt{setZoom(zoom : int) : void}
        \\ Setzt den Zoom der Leaflet-Karte auf \emph{zoom}.
    \end{itemize}
    \textbf{Attribute}
    \begin{itemize}
        \item \texttt{viewport : Viewport}
        \\ Der aktuelle Viewport der Leaflet-Karte
    \end{itemize}
\end{Class}

\begin{Class}{Legend extends React.Component}
    \textbf{Attribute}
    \begin{itemize}
        \item \texttt{scale : Skala}
        \\ Die aktuell verwendete Skala.
    \end{itemize}
\end{Class}

\begin{Class}{StationInfo extends React.Component}
    \textbf{Attribute}
    \begin{itemize}
        \item \texttt{openDetails : MaterialUI.Button}
        \\ click() öffnet die Detailansicht mit der ID der \emph{station}
        \item \texttt{station : Messstation}
        \\ Die Messstation zu der die StationInfo gehört.
        \item \texttt{isOpen : boolean}
        \\ \emph{true}: StationInfo wird bei der Position der \emph{station} angezeigt
        \\ \emph{false}: StationInfo ist ausgeblendet
    \end{itemize}
\end{Class}

\subsection{MapPage}

\begin{Class}{Viewport}
    \textbf{Methoden}
    \begin{itemize}
        \item \texttt{getCenter() : Position}
        \\ Gibt das aktuelle Zentrum zurück.
        \item \texttt{setCenter(position : Position) : void}
        \\ Setzt \emph{position} als das Zentrum fest.
        \item \texttt{getZoom() : int}
        \\ Liefert den Zoom des Viewports.
        \item \texttt{setZoom(zoom : int) : void}
        \\ Setzt den Zoom des Viewports auf \emph{zoom}.
        \item \texttt{getRadius() : int}
        \\ Liefert den kleinsten Radius des Kreises in Koordinateneinheiten um das Zentrum zurück,
         der den kompletten Anzeigebereich des Viewport einschließt.
    \end{itemize}
\end{Class}

\begin{Class}{Position}
    \textbf{Methoden}
    \begin{itemize}
        \item \texttt{getCoordinates() : (lng : float, lat : float)}
        \\ Liefert die Koordinaten der Position als Tupel.
        \item \texttt{setCoordinates(lng : float, lat : float)}
        \\ Setzt die Koordinaten auf lng und lat fest.
        \\ Ist einer der beiden Parameter \emph{null} bleibt dieser unverändert, der andere wird angepasst.
        \item \texttt{toString() : string}
        \\ Liefert die Position als string im Dezimalgrad-Format.
    \end{itemize}
\end{Class}

\begin{Class}{<<abstract>> MapConfiguration}
    \textbf{Methoden}
    \begin{itemize}
        \item \texttt{getPins(port : Viewport) : MapPin[]}
        \\ Liefert MapPins innerhalb (oder nahe) des Viewports zurück.
        \item \texttt{getPolygons(port: Viewport) : Polygon[]}
        \\ Liefert Polygone innerhalb (oder nahe) des Viewports zurück.
        \item \texttt{getScale() : Skala}
        \\ Liefert die Skala der Konfiguration zurück.
        \item \texttt{getFeatures() : Feature[]}
        \\ Liefert das Features die die Konfiguration darstellt zurück.
        \\ \emph{[]}, wenn keine Features dargestellt werden
    \end{itemize}
\end{Class}

\begin{Class}{StationConfiguration extends MapConfiguration}
    Eine Kartenkonfiguration die die Stationen mit Farbe nach einer Skala anzeigt.
    \textbf{Methoden}
    \begin{itemize}
        \item \texttt{setFeature(feature : Feature) : void}
        \\ Legt das Feature der Konfiguration fest.
        \item \texttt{<<implements>>getPins(port : Viewport) : MapPin[]}
        \\ Liefert MapPins innerhalb (oder nahe) des Viewports zurück.
        \\ Es werden nur Stationen angezeigt die das ausgewählte Feature besitzen
        \item \texttt{<<implements>>getPolygons(port: Viewport) : Polygon[]}
        \\ Liefert immer [] zurück.
        \item \texttt{<<implements>>getScale() : Skala}
        \\ Liefert die Skala des gewählten Features zurück.
        \item \texttt{<<implements>>getFeatures() : Feature[]}
        \\ Liefert das ausgewählte Feature in einem Array zurück.
    \end{itemize}
\end{Class}

\begin{Class}{MapPin}
    \textbf{Methoden}
    \begin{itemize}
        \item \texttt{getPosition() : Position}
        \\ Liefert die Position des Pins
        \item \texttt{setPosition(position : Position) : void}
        \\ Setzt die Position des Pins

        \bigskip
        \item \texttt{getValue() : float}
        \\ Liefert den Wert zurück anhand dem die Färbung bestimmt wird.
        \item \texttt{setValue(value : float) : void}
        \\ Setzt den Wert des Pins.

        \bigskip
        \item \texttt{getStationId() : string}
        \\ Liefert die interne Stations ID der Station die der Pin repräsentiert.
        \item \texttt{setStationId(id : string) : void}
        \\ Setzt die interne Stations ID der Station die der Pin repräsentiert.

        \bigskip
        \item \texttt{getColor() : Color}
        \\ Liefert die Farbe des Pins.
        \item \texttt{setColor(color : Color) : void}
        \\ Legt die Farbe des Pins fest.
    \end{itemize}
\end{Class}

\begin{Class}{Polygon}
    \textbf{Attribute}
    \begin{itemize}
        \item \texttt{stations : ObservationStation[]}
        \\ Liefert die Stationen zurück die die Ecken des Polygons bilden.
    \end{itemize}
\end{Class}