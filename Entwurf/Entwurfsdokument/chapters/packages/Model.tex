\subsection{Model}

    \begin{Class}{ObservationStation}
        Eine Messstation mit Metadaten, verfügbaren Features und Diagrammen.
        \textbf{Methoden}
        \begin{itemize}
            \item \texttt{constructor(id: string, name: string,
            \\ desc : string, pos: Position)}
            \\ Erstellt eine ObservationStation mit den angegebenen Attributen.
            Wirft einen Fehler wenn \emph{id} nicht das Format einer iot-id hat.
            \item \texttt{setFeatures(features : Features[])}
            \\ Setzt die unterstützten Features auf \emph{features}.
            \item \texttt{getId() : string}
            \\ Die iot-id der Station.
            \item \texttt{getName() : string}
            \\ Der Name der Station.
            \item \texttt{getDescription() : string}
            \\ Die Beschreibung der Station in der in \emph{Language} ausgewählten Sprache.
            \item \texttt{getFeatures() : Feature[]}
            \\ Array von Features, die die Station erfassen kann, und von der Anwendung unterstützt werden.
            \item \texttt{getPosition() : Position}
            \\ Die Position an der sich die Station befindet.
            \item \texttt{async getDiagrams() : Promise<DetailPage.Diagram[]>}
            \\ Liefert ein Promise auf ein Array an Diagrammen.
            Nach der Erfüllung werden die unterstützen Diagramme der verfügbaren Features ausgegeben.
            \item \texttt{hasFeature(feature : Feature) : boolean}
            \\ \emph{true} wenn die Station das Feature erfassen kann.
            \\ \emph{false} sonst.
        \end{itemize}
    \end{Class}

    \begin{Class}{Observation}
        Ein Messwert eines bestimmten Features an einer Messstation zu einem gewissen Zeitpunkt.
        \textbf{Methoden}
        \begin{itemize}
            \item \texttt{constructor(observationStation: ObservationStation, 
            \\feature: Feature, date: Date, value: number)}
            \\ Erstellt eine Observation mit den angegebenen Attributen.
            \item \texttt{getObservationStation() : ObservationStation}
            \\ Die Messstation an der die Messung durchgeführt wurde.
            \item \texttt{getFeature() : Feature}
            \\ Gibt das Feature der Beobachtung zurück.
            \item \texttt{getTimeStamp() : Date}
            \\ Gibt den Zeitpunkt der Beobachtung zurück.
            \item \texttt{getValue() : number}
            \\ Gibt den Wert der Beobachtung (in der Einheit des Feature) zurück.
        \end{itemize}
    \end{Class}

    \begin{Class}{Position}
        Eine Position auf der Welt in Koordinaten.
        \textbf{Methoden}
        \begin{itemize}
            \item \texttt{constructor(latitude: number, longitude: number)}
            \\ Erstellt eine Position an \emph{latitude} °N, \emph{longitude} °O
            \item \texttt{getCoordinates(): [number, number]}
            \\ Die Koordinaten der Position als [lat, lng] Tupel.
            \item \texttt{getLatitude(): number}
            \\ Die Latitude der Position.
            \item \texttt{getLongitude(): number}
            \\ Die Longitude der Position
            \item \texttt{toString() : string}
            \\ Gibt die Daten im Format 'N 0.00000°, O 0.00000°' als Zeichenkette zurück.
        \end{itemize}
    \end{Class}

    \begin{Class}{Feature}
        Eine Umwelteigenschaft die von einer Messstation gemessen werden kann, Metadaten
        und mögliche Diagramme.
        \textbf{Methoden}
        \begin{itemize}
            \item \texttt{constructor(id : string, nameId : string, descriptionId : string,\\
             scale : Scale, relatedWeblink : string,
            \\drawableDiagrams : DetailPage.Diagram[],
            \\limit: number, unitOfMeasurement: string)}
            \\ Erstellt ein Feature mit den angegebenen Attributen.
            Wirft einen Fehler wenn \emph{scale} keine Farbstufen hat.
            \item \texttt{getId() : string}
            \\ Gibt die FROST-Id des Features zurück.
            \item \texttt{getName() : string}
            \\ Der lokalisierte Featurename in der aktuellen Sprache.
            \item \texttt{getUnitOfMeasurement() : string}
            \\ Gibt die Einheit in der das Feature angegeben ist zurück.
            \item \texttt{getDescription() : string}
            \\  Die lokalisierte Beschreibung in der aktuellen Sprache.
            \item \texttt{setScale(scale: Scale)}
            \\ Setzt die Farbskala auf \emph{scale}.
            \item \texttt{getRelatedScale() : Scale}
            \\ Die Farbskala des Features.
            \item \texttt{getRelatedWeblink(): string}
            \\ Eine URL zu einer Definition / Erklärung des gemessenen Wertes.
            \item \texttt{getLimit() : number}
            \\ Das Limit für eine Warnmeldung. Wenn das Limit -1 ist, wird nie eine Warnung ausgegeben.
            \item \texttt{isLimitExceeded(obs : Observation) : Boolean}
            \\ \emph{true} wenn der Wert der Observation das Limit des Features übersteigt.
            \\ \emph{false} wenn das Limit -1 ist oder der Wert der Observation <= Limit.
        \end{itemize}
    \end{Class}

    \begin{Class}{Color}
        \textbf{Methoden}
        \begin{itemize}
            \item \texttt{constructor(r: number, g: number, b: number)}
            \\ Erstellt eine Farbe aus den RGB Werten.
            \item \texttt{constructor(hex: string)}
            \\ Erstellt eine Farbe aus dem Hexcode im Format '\#FFFFFF'.
            Entspricht \emph{hex} nicht dem Format wird ein Fehler geworfen.
            \item \texttt{getRGB() : [number, number, number]}
            \\ Die RGB Werte der Farbe als Zahlentripel.
            \item \texttt{getHex() : string}
            \\ Liefert den RGB-Wert der Farbe im Format '\#FFFFFF' zurück.
        \end{itemize}
    \end{Class}

    \begin{Class}{Scale}
        \textbf{Methoden}
        \begin{itemize}
            \item \texttt{constructor(linearTransition: boolean,
            \\colors : {number, string}[])}
            \\ Erstellt eine Skala entsprechend der Attribute.
            Ist \emph{colors} leer wird ein Wert mit \{0, "\#BBBBBB"\} erstellt.
            \item \texttt{getColor(value : number) : Color}
            \\ Gibt die Farbe des Wertes entsprechend der Konfiguration zurück.
            Ist \emph{linearTransition} True wird die Farbe als Durchschnitt der zwei nächsten Farben berechnet.
        \end{itemize}
    \end{Class}

    \begin{Class}{Polygon}
        \textbf{Attribute}
        \begin{itemize}
            \item \texttt{stations : ObservationStation[]}
            \\ Stationen, die die Ecken des Polygons bilden.
            \item \texttt{color : Color}
            \\ Farbe des Polygons.
        \end{itemize}
    \end{Class}

    \begin{Class}{MapPin}
        \textbf{Methoden}
        \begin{itemize}
            \item \texttt{getPosition() : Position}
            \\ Liefert die Position des Pins
            \item \texttt{setPosition(position : Position) : void}
            \\ Setzt die Position des Pins

            \bigskip
            \item \texttt{getValue() : float}
            \\ Todo: Genaue Dokumentation für value.
            \\ Liefert den Wert anhand dem die Färbung bestimmt wird.
            \item \texttt{setValue(value : float) : void}
            \\ Setzt den Wert des Pins.

            \bigskip
            \item \texttt{getStationId() : string}
            \\ Liefert die interne Id der Station, die durch den Pin repräsentiert wird.
            \item \texttt{setStationId(id : string) : void}
            \\ Setzt die interne Id der Station, die durch den Pin repräsentiert wird.

            \bigskip
            \item \texttt{getColor() : Color}
            \\ Liefert die Farbe des Pins.
            \item \texttt{setColor(color : Color) : void}
            \\ Legt die Farbe des Pins fest.
        \end{itemize}
    \end{Class}

    \begin{Class}{Viewport}
        \textbf{Methoden}
        \begin{itemize}
            \item \texttt{getCenter() : Position}
            \\ Liefert das Zentrum des Sichtfensters 
            \item \texttt{getZoom() : number}
            \\ Liefert das Zoomlevel des Sichtfensters
            \item \texttt{getRadius() : number}
            \\ Liefert den Durchmesser des Sichtfensters, 
            d.h. der Kreis um den Mittelpunkt der das ganze Sichtfenster einschließt.
            \item \texttt{setCenter(position : Position) : void}
            \\ Setzt das Zentrum des Fensters auf \emph{position}.
            \item \texttt{setZoom(zoom : number) : void}
            \\ Setzt das Zoomlevel auf \emph{zoom} wenn \emph{zoom} > 0.
        \end{itemize}
    \end{Class}