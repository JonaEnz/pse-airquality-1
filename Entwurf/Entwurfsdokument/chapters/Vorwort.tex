\section{Vorwort}
Dieses Dokument erklärt die Systemarchitektur unseres PSE Projekts: "Anwenderorientierte Nutzerschnittstelle für Luftqualitätsdaten". Ziel des Projekt ist es, eine Webanwendung zu entwickeln, die Daten aus der \gls{SmartAQnet}-Datenbank, in einer auf den Nutzer zugeschnittenen Art und Weise, visualisiert und aufbereitet.

Wir haben uns dafür entschieden die Webanwendung als reine Frontend-Anwendung mit dem Webframework React und unter der Programmiersprache Typescript zu entwickeln, mit dem Ziel, das Programm stabil und objektorientiert zu gestalten. Der Komponentenbasierte Entwurf mit React und die Typsicherheit unter Typescript vereinfachen nicht nur den Entwicklungsprozess, sondern bieten auch Flexibilität und Skalierbarkeit.

Bei der groben Systemarchitektur haben wir uns an dem "Model, View, Controller" Architekturmodell orientiert. Diese Entwurfsentscheidung wird im Kapitel "Architektur" weiter beschrieben und erklärt.

Des Weiteren verwenden wir entsprechend den Vorgaben von PSE einen objektorientierten Entwurf. Die Aufteilung und Struktur der einzelnen Klassen und Komponenten wird im Kapitel "Klassenbeschreibung" und in den dazugehörigen UML-Klassendiagrammen beschrieben.
Das Zusammenspiel der Klassen in einem Anwendungsszenario haben wir in einigen Sequenzdiagrammen ausgeführt.

Insgesamt soll dieses Dokument einen umfassenden Überblick über den geplanten Systementwurf geben und somit als Basis für die anschließende Implementierung der Anwendung dienen.

