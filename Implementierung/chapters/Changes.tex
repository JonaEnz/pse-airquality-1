\section{Änderungen}
\begin{Change}{FeatureProvider}
    Die Funktionalität dieser Klasse sollte ursprünglich in 
    Controller.Frost.DataProvider enthalten sein und nach außen versteckt. 
    Da die Features aus den Konfigurationsdateien allerdings logisch nicht 
    vom Server abhängen wurden sie ausgelagert. Dies ermöglicht außerdem den 
    Zugriff auf die Features auch außerhalb des FROST-Pakets.
    
    \bigskip
    \textbf{Methoden}
    \begin{itemize}
        \item \texttt{static getInstance()}
        \\ Liefert die Singleton-Instanz zurück oder erstellt sie.
        \item \texttt{constructor()}
        \\ Initialisiert den Feature-Speicher
        \item \texttt{getFeature(id : string) : Feature|undefined}
        \\ Das gespeicherte Feature falls es bereits geladen wurde.
        Sonst wird zunächst das Feature aus der Konfigurationsdatei geladen.
        Schlägt dies fehl wird 'undefined' zurückgegeben.
    \end{itemize}
\end{Change}

\begin{Change}{MapController}
    
    Skala und Viewport werden von außen lesbar gemacht.
    Damit muss MapPage keine eigene Kopie bereithalten.

    \bigskip
    \textbf{Hinzugefügt}
    \begin{itemize}
        \item \texttt{getScale() : Scale}
        \\ Die aktuelle Skala.
        \item \texttt{getViewport(): Viewport}
        \\ Der aktuelle Viewport.
    \end{itemize}

\end{Change}

\begin{Change}{OnSearch(string) für Suche}
    
    Die eigentliche Suche findet nun außerhalb der Komponente statt.
    Damit wird die Komponente leichter für verschiedene Anwendungen wiederverwendbar.

    \bigskip
    \textbf{Hinzugefügt}
    \begin{itemize}
        \item \texttt{Search.Props.onSearch(term: string) : void}
        \\ Wird bei Klick auf den Such-Button oder Drücken von Enter aufgerufen.
        Enthält den aktuellen Inhalt der Suchbox. 
        \item \texttt{MapView.onSearch(term: string) : void}
        \\ Ruft die Suche im MapController auf und aktualisiert die Seite.
    \end{itemize}

\end{Change}

\begin{Change}{Legend}
    
    Der Ausschnitt der von der Legende angezeigt wird soll flexibel sein.

    \bigskip
    \textbf{Hinzugefügt}
    \begin{itemize}
        \item \texttt{Props.min : number}
        \\ Das untere Ende der Legende
        \item \texttt{Props.max : number}
        \\ Das obere Ende der Legende
    \end{itemize}

\end{Change}

\begin{Change}{Map}
    
    Ursprünglich sollte der Viewport über die Mitte der Pins/Polygone bestimmt werden.
    Um die Karte von Anfang an auf die letzte Position zu zentrieren wird der Viewport direkt übergeben.
    
    \bigskip
    \textbf{Hinzugefügt}
    \begin{itemize}
        \item \texttt{Props.viewport : Viewport}
        \\ Viewport mit dem die Karte initialisiert wird.
    \end{itemize}

\end{Change}

\begin{Change}{FeatureSelectInit}
    
    Die FeatureSelect Auswahl benötigt die aktuellen Werte um sie standardmäßig auswählen zu können.
    
    \bigskip
    \textbf{Hinzugefügt}
    \begin{itemize}
        \item \texttt{FeatureSelect.Props.startConf?: \{ conf: string; feature: string \}}
        \\ Die Startwerte der Auswahlboxen. Optional.
        \item \texttt{MapController.getFeatureSelectConf(): { conf: string; feature: string }}
        \\ Gibt Werte für die Auswahlboxen basierend auf der Konfiguration aus.
    \end{itemize}

\end{Change}