\section{Herausforderungen}

\subsection{MapConfigurationMemory}

Diese Klasse erfüllt die Aufgabe die aktuell gewählte Konfiguration und Viewport in den 
Local Storage zu schreiben und später wieder zu laden. Bei der testweisen Implementierung während
der Entwurfsphase gab es hierbei keine Probleme. Da dort aber lediglich mit Interfaces als Speicherdaten
gearbeitet wurde und nun Klassen vorhanden sind ergaben sich neue Schwierigkeiten.
So können die Objekte nun nicht mehr nur mit der JSON Klasse de-/serialisiert werden, sondern müssen
vom Programm erstellt werden und mit einzelnen Werten aus dem \texttt{Local Storage} befüllt werden.
Da dies leichte Anpassungen im \texttt{MapController} erforderte verzögerte sich hier der Zeitplan 
geringfügig.

\subsection{FeatureProvider}

Hier führte eine Eigenheit von React / Webpack zu erheblichen Verzögerungen: 
require('...') kann nur mit festen Zeichenketten problemlos funktionieren sofern die benötigten
Dateien nicht im selben Ordner liegen. Nach einiger Recherche werden die Definitionen nun mit 
require.context(...) geladen, was allerdings einen großen Nachteil mit sich bringt.
Jest unterstützt diesen Befehl nicht was dazu führt dass alle Tests abstürzen die irgendein Feature
benutzen wollen.
Da das gerade mit Hinblick auf die Qualitätssicherungs-Phase nicht tragbar ist wurden letztendlich
alle Features in einer feature.json als Liste gespeichert und beim Anwendungsstart geladen.
So wird nur wenig Flexibilität eingebüßt.