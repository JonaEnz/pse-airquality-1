\section{CI/CD}

\subsection{Vorteile}
Die verwendeten Services stellen sicher dass die Tests und Builds immer unter genau definierten Bedingungen
ablaufen und nicht von der speziellen Konfiguration des Rechenes abhängen auf dem entwickelt wird.
Fehlerhafte Commits werden automatisch markiert und die Fehlermeldung ist ersichtlich ohne den Code selbst
auszuführen.

\subsection{Automatisiertes Testen}

Mithilfe von Azure Pipelines werden nach jedem Commit auf den master-Zweig die 
Unit-Tests ausgeführt 
und der Autor des Commits per Mail benachrichtigt. Zudem wird die Code-Abdeckung geprüft und eine Übersicht
aus der Cobertura Datei erstellt. Dadurch kann der zeitliche Verlauf der Abdeckung zurückverfolgt und Lücken
schnell identifiziert werden. \\
Die Ergebnisse des letzten Runs werden zudem direkt auf GitHub angezigt.\\
\url{https://jonaenzinger.visualstudio.com/pse-airquality-react/_build?definitionId=5}

\subsection{Automatisierte Auslieferung}

Nach jeder akzeptierten Pull-Request oder Commit auf den master-Zweig wird versucht einen Build der Seite
zu erstellen. Ist dies erfolgreich und schlagen keine Unit-Tests fehl wird das Ergebnis des Build
automatisch auf GitHub Pages veröffentlicht. So muss der Auslieferungsvorgang nicht für jeden Recher
einzeln konfiguriert werden sondern kann zentral mit Bedingungen wie auf dem Webserver durchgeführt werden.