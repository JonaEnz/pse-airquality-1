\section{Usability Test}
Bei den Usability Tests wird das fertige Produkt an echten Testpersonen getestet, deren  Meinungen und Reaktionen erfasst und analysiert werden. Dadurch wird die Benutzerfreundlichkeit und intuitive Bedienenug des Produktes gemessen. Diese Test sind essenziell, da das Entwicklerteam, das schon von anfang an an das Projekt arbeitet die Funktionalität des Programms vollständig kennt und deshalb die Intuitivität (Bedienung ohne jegliche Kenntnis des Programms) nicht beurteilen kann.

  \subsection{Testpersonen}
    Die Testpersonen sind - wie unsere Zielgruppe - eher älter, wobei keine genaue untere Grenze gesetzt ist und die Meinungen der Testpersonen außerhalb unserer Zielgruppe gleich wertvoll für uns sind. Es ist wichtig, dass die Testsubjekte über wenig bis gar kein Wissen über Feinstaubindex bzw. Luftqualität verfügen, da die Applikation insbesondere für solche Menschen entwickelt wurde. Personen, die wenig über Softwareentwicklung wissen, werden bevorzugt. Die Testsubjekte hauptsächlich Bekannten des Entwicklerteams, da die Applikation nicht als Webseite zugänglich ist und weil für den Test die Beobachtung, Erklärung und Fragestellung eines Teammitglieds mit breitem Wissen über die Applikation wichtig ist.

  \subsection{Testmethode}
    Um ein umfassendes Bild über die Usability unserer Webanwendung zu bekommen haben wir einen umfangreichen Usability-Test entworfen. Der Test besteht im wesentlichen aus 3 Teilen, die im folgenden noch genauer beschrieben werden. Den kompletten Test finden Sie im Anhang.
    Wir haben den Test mit jeder Testperson einzeln, an verschiedenen Geräten (PC/Laptop/Tablet/Smartphone) durchgeführt.

    \subsubsection*{System Usability Scale}
      Durch das System Usability Scale wird die subjektiv wahrgenommene Gebrauchstauglichkeit des Programs bewertet. Die Testsubjekte müssen zehn Fragen mit Werten von 1 bis 5 bewerten. Wir haben unsere Fragen aus Wikipedia und aus den Fragen der SAP genommen. Diese Methode an sich gibt keinen genauen Überblick über die Probleme mit dem Programm, weshalb wir zusätzlich noch Think Aloud verwendet haben.
      
    \subsubsection*{Think Aloud}
      Beim Teil 'Think ALoud' des Usability Tests wurden den Testpersonen einfache Frage und Aufgaben gestellt. Die Tester wurden angewießen die die Fragen intuitiv zu beantworten und beim Lösen der Aufgaben pragmatisch vorzugehen und ihre Gedankengänge dabei laut auszusprechen.
    
    \subsubsection*{Erweitereter Fragebogen}
      Hier haben wir vor allem Freitext-Fragen abgefragt. Unter anderem war die Frage, was den Testpersonen am meisten/wenigsten an der Webanwendung gefallen hat und welche Verbesserungen sie uns empfehlen würden.

  \subsection{Auswertung}
    Da die Antworten unserer Testpersonen sehr umfangreich und teilweise sehr unterschiedlich oder auch teilweise unrelevant waren, haben wir uns im folgenden auf eine Zusammenfassung der Ergebnisse beschränkt.

    \subsubsection{Think Aloud}
      \paragraph{Kartenansicht}
        Der erste Eindruck der Tester zur Kartenansicht war in 9 von 10 Fällen positiv. Viele erwähnten, dass sie das Design übersichtlich und intuitiv fänden. Zwei Tester zogen einen Vergleich zu anderen Kartenanwendungen.

        Anschließend haben wir einige Aufgaben gestellt, um zu sehen, wie die Tester intuitiv vorgehen, um verschiedene Orte auf der Karte zu finden. Hiebei viel auf, dass vor allem (in 13 von 20 Fällen) die Suchfunktion genutzt wurde. Die anderen Tester zoomten heraus, verschoben die Karte und zoomten an der gewünschten Stelle wieder herein.

        Anschließend versuchten wir mit einigen Fragen herauszufinden, ob die Darstellung der Messstationen als Marker auf der Karte intuitiv ist. Alle Tester bestätigten das. Spätestens nach dem anklicken eines Markers nannten alle Tester die richtige Bedeutung der Marker und der dargestellten Daten.

        Die nächsten Aufgaben zielten darauf ab, die Benutzbarkeit der Kartenkonfiguration zu testen. Die Tester wurden angewießen verschiedene Einstellungen in der Webapp zu finden. Unter anderem sollte der Temparatur Höchstwert innerhalb Augsburg ermittelt werden. Mit der Standarteinstellung war das zwar möglich aber sehr umständlich, da bei ähnlichen Temparaturen die Marker ähnlich eingefärbt waren, war das Erkennen von Unterschieden erschwerte.
        Einfacher hätte man die Aufgabe durch eine Umstellung der Karteneinstellung 'Style' hätte lösen können woraufhin die Messwerte nicht mehr auf einer absoluten Skala eingefärbt würden, sondern im Vergleich zu den umliegenden Stationen. Diese Einstellung wurden aber von keiner Testperson vorgenommen. Auf Nachfrage wurde angegeben, das der Begriff 'Style' in diesem Zusammenhang missverständlich sei.
     
      \paragraph{Detailansicht}
        Der erste Eindruck der Detailansicht war in allen Fällen positiv. Auch hier wurde der Aufbau der Seite als übersichtlich und schlicht beschrieben.
        Die Bedeutung des Diagramms, das die zeitliche Veränderung der Messwerte eines \glspl{Feature} anzeigt war allen Testpersonen intuitiv klar, im Gegensatz zur Bedeutung des Tortendiagramms, dessen Bedeutung einigen Testpersonen unklar war.
        
        Die Konfiguration der Diagramme war für alle Testpersonen intuitiv möglich.
     
    \subsubsection{Auswertung System Usability Scale}
      \includegraphics[width=1\linewidth]{figures/SUSscore.png}\par\vspace{1cm}

    \subsubsection{Auswertung Erweiterter Fragebogen}
      \paragraph{Positives}
        Mehrmals wurde hier das schlichte Design der Webanwendung angegeben, das viele Tester als übersichtlich und intuitiv beschrieben haben.
        Des Weiteren wurde gelobt, dass die Webanwendung viele nützliche Informationen bereit stellt und auch Detailinformationen erreichbar sind.
        Einige Tester lobten die Einordung der Messwerte mit hilfe der Farbskala.
     
      \paragraph{Negatives und Verbesserungsvorschläge}
        Bei der Frage, welche Punkte den Testpersonen nicht an unserer Webanwendung gafallen haben, gab es bis auf einen Punkt keinen eindeutigen Trend, da die Anworten sehr unterschiedlich waren.
        Ein Tester beschrieb die Webanwendung als 'zu schlicht' und hätte sich ein auffallenderes Design und eine andere Hauptfarbe gewünscht.
        Zwei Tester bemängelten, dass nur Messstationen in Augsburg unterstützt würden.
        Ein weiterer Tester gab an, die Farben der Farbskala der Kartenansicht seien zu änlich, wodurch die Unterschiede zwischen den Messwerten unterschiedlicher Messstationen auf der Karte nur sehr schwer zu erkennen seien.
        
        Der Punkt, in dem sich die meisten Tester einig waren, war der, das die Erklärungen der Fachbegriffe (vor allem PM10 und PM2.5) nur schwer in der Webapp zu finden seien. Diese Hilfe-Funktion solle auffallender in der Webapp positioniert werden.

  \subsection{Gelöste Probleme}
    \subsubsection*{Konfigurationsnamen}
      \paragraph{Problem}
      Bei der Auswahl der Konfiguration wird nur der interne Name angezeigt (z.B. PolygonConfiguration),
      was wenig aussagekräftig ist. Zudem sind diese Namen nicht lokalisiert.

      \paragraph{Lösung}
      Die IDs der Konfigurationen bekommen einen Eintrag in der Sprachdatei um leicht verständliche
      Namen in der gewählen Sprache anzuzeigen.

    \subsubsection*{Weitere Optimierung der Detailseite für Mobilgeräte}
      \paragraph{Symptom}
        Auf Mobilgeräten mit besonders kleinen Bildschirmgrößen sind die Diagramme der Detailseite zu klein um Details zu erkennen.

      \paragraph{Grund}
      Bislang gab es noch kein optimiertes Design dieser Seite für kleine Mobilgeräte.

      \paragraph{Lösung}
      Die Breakpoints des Grid Layouts wurden angepasst, sodass die Diagramm auf Mobilgeräten nun untereinander, anstatt in Zweierreihen angezeigt werden.

    \subsubsection*{FeautureSelect-Button}
      \paragraph{Symptom}
      Das FeatureSelect-Button ist für Personen, die die Webanwendung zum ersten Mal benutzen, schwer zu finden.  

      \paragraph{Grund}
      Das Button ist zu klein, lässt sich schwer vom Hintergrund unterscheiden. Man hat auch keine Information was es macht bis man draufklickt.

      \paragraph{Lösung}
      Das FeatureSelect-Menü ist bei Laden der Seite geöffnet und somit für Benutzer leicht zu finden.

    \subsubsection*{Skala}
      \paragraph{Symptom}
      Größter Wert der Skala ist unrealistisch groß und damit ist die ganze Skala nutzlos.

      \paragraph{Grund}
      Die Skala ist auf den extremen Messwerten aller Messstation festgelegt. Da eine Messstation falsche Werte sendet, wird die Skala falsch eingestellt.

      \paragraph{Lösung}
      Entsprechende Station aus den Anfragen entfernen.

    \subsubsection*{Suche}
      \paragraph{Symptom}
      Der Platzhaltersatz bei der Suche ist nicht vollständig lesbar.

      \paragraph{Grund}
      Feld  ist zu kurz.

      \paragraph{Lösung}
      Kürzerer Platzhaltertext. Die Größe des Feldes ist für normaler Städtenamen mehr als ausreichend.

    \subsubsection*{Vergleich mit letztem Jahr}
      \paragraph{Symptom}

      Diagramm enthält überhaupt keine Information, was es zeigen will. Der Nutzer kann das Diagramm ohne Erklärung der Entwickler nicht verstehen.


      \paragraph{Grund}
      Überschrift und Beschriftung der Achsen fehlen.

      \paragraph{Lösung}
      Überschrift für Diagramme hinzugefügt.

    \subsubsection*{Polygon}
      \paragraph{Symptom}

      Man sieht zwar die Färbung der Polygonen, aber man kann ihnen keinen Wert zuordnen.


      \paragraph{Grund}
      Durchschnittswert der Polygonen wird nirgendwo angezeigt.

      \paragraph{Lösung}
      Der Durchschnittswert wird in einem Tooltip angezeigt. Dafür musste die Polygonklasse geringfügig
      angepasst werden. Sie verwendet nun Observations statt Stations, was eine sinnvolle Erweiterung für die Zukunft ist.


    \subsubsection*{Namen der Stationen}
      \paragraph{Symptom}

      Namen und Koordinaten der Stationen sind für die Meisten uninteressant und nichtssagend.
      

      \paragraph{Lösung}
      Zusätzlich zu den Koordinaten und dem Stationnamen haben wurde auch noch die Straße hinzugefügt.

    \subsubsection*{Fragezeichen nur auf in der Kartenansicht}
      \paragraph{Symptom}
      Bei den Stationen-PopUps gibt es ein Fragezeichen-Button, die auf eine Webseite mit mehr Informationen über die ausgewählten Features weiterführt. Es wäre praktisch, diesr Fragezeichen auch auf der detailseite zu haben, insbesondere weil dort alle Features nebeneinander stehen und so man sich alle nacheinender anschauen kann.

      \paragraph{Lösung}
      Fragezeichen-Button mit gleicher Funktion zu jedem angezeigten Feature auf der Detailseite wurde hinzugefügt.

    \subsubsection*{Weitere Informationen - Größe und Stelle der Fragezeichen}

      \paragraph{Symptom}
      Einige Testsubjekte fanden das Fragezeichen einfach zu übersehen und dessen Stelle unpraktisch. Nach mehr Testdurchläufen mit neuen Testsubjekten haben wir festgestellt, dass nur ein kleiner Teil der Testsubjekten das Fragezeichen problematisch gefunden hat. Einige wurden sogar extra von uns darauf angesprochen, ihre Meinung zum Fragezeichen-Button uns mitzuteilen. Sie hatten entweder keine Meinung dazu oder fanden es sogar gut.

      \paragraph{Lösung}
      Wir haben eine Anleitung zur Webseite hinzugefügt, und deren Button an eine auffällige Stelle platziert. In der Anleitung gibt es auch einen Kapitel über das Fragezeichen-Button mit Illustration. 


    \subsection*{Unwichtige\textbackslash Unlösbare Probleme}

      In diesem Kapitel listen wir die Probleme auf, die wie versucht haben zu lösen, aber die Lösung wäre entweder zu aufwendig gewesen oder hätte nicht in unseren Zeitplan gepasst.


    \subsubsection*{Sichtbarkeit der Stadtname}
      \paragraph{Symptom}
      Die Pins bedecken den Stadtnamen. Er ist bei vielen Pins nicht mehr lesbar.

      \paragraph{Grund}
      Ortbeschriftungen befinden sich auf der Ebene unter den Pins. 

      \paragraph{Lösung}
      Die Ortsnamen sind direkt in die Kachelbilder eingetragen was es unmöglich macht sie 
      auf eine andere Ebene zu verschieben. Eine Option wäre Reverse Geocoding.