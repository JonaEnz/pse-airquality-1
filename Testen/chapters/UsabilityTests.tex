\section{Usability tests}

Bei den Usability Tests wird das fertige Produkt an echten Testpersonen getestet, deren  Meinungen und Reaktionen erfasst und analysiert werden. Dadurch wird die Benutzerfreundlichkeit und intuitive Bedienenug des Produktes gemessen. Diese Test sind essenziell, da das Entwicklerteam, das schon von anfang an an das Projekt arbeitet die Funktionalität des Programms vollständig kennt und deshalb die Intuitivität (Bedienung ohne jegliche Kenntnis des Programms) nicht beurteilen kann.

\subsection{Testsubjekte}

Die Testsubjekte sind - o wie die Zielgruppe der Umfrage im Pflichtenheft- eher älter, wobei keine genaue untere Grenze gesetzt ist und die Meinungen der Subjekte außerhalb der Zielgruppe uns gleich wertvoll ist. Es ist wichtig, dass die Testsubjekte über wenig bis gar kein Wissen über Feinstaubindex bzw. Luftqualität verfügen, da die Applikation insbesondere für solche Menschen entwickelt wurde. Personen, die wenig über Softwareentwicklung wissen, werden bevorzugt. Die Testsubjekte sind entweder Bekannten der Mitglieder des Entwicklerteams- sie werden persönlich getestet- oder Fremde -für diesen Fall verstellen wir eine Online-Umfrage. 

\subsection{Testmethoden} 
\subsubsection{Zielgruppe}
\subsubsection{System Usability Scale}
\subsubsection{Hallway Tests}
\subsubsection{Responsiveness}

\begin{itemize}
  \item Die Anforderungen, die zu Wunschkriterien gehören, werden mit * gekennzeichnet.
\end{itemize}

\subsection{Bugs and Fixes}

\subsubsection{Konfigurationsnamen}
\paragraph{Problem}
Bei der Auswahl der Konfiguration wird nur der interne Name angezeigt (z.B. PolygonConfiguration),
was wenig aussagekräftig ist. Zudem sind diese Namen nicht lokalisiert.

\paragraph{Lösung}
Die IDs der Konfigurationen bekommen einen Eintrag in der Sprachdatei um leicht verständliche
Namen in der gewählen Sprache anzuzeigen.

\subsubsection{Diagramme auf Mobilgeräten}
\paragraph{Symptom}
Auf Mobilgeräten werden in der Detailansicht wie auf dem Desktop jeweils zwei Diagramme nebeneinander angezeigt.
Das macht sie jedoch kaum lesbar.

\paragraph{Grund}
Die Anzeige orientiert sich an einem Grid-Element das keine automatische Anpassung auf 
die Bildschirmgröße bereitstellt.

\paragraph{Lösung}
!TODO

\subsubsection{Datenmenge für Diagramme}
\paragraph{Symptom}
Durch den Entwurf der Anwendung als SPA die komplett beim Nutzer läuft müssen für die Diagramme teils große
Datenmengen angefragt werden. So benötigt eine normale Detailseite bis zu 60 MB an Mess- und Seitendaten um
angezeigt zu werden. Gerade dür die Nutzung auf getakteten Mobilfunkverbindungen ist das für den Nutzer
unerwartet und inakzeptabel

\paragraph{Grund}
Viele Sensoren messen alle 5 Minuten was in einem Jahr schon ~100.000 Messwerte sind.
Somit erhöhen vor allem Diagramme die längerfristige Trends betrachten die Datenmenge enorm.

\paragraph{Lösung}
Es gibt grundsätzlich 3 verschiedene Ansätze das Problem abzuschwächen:
\begin{itemize}
  \item Erstellung eines API-Servers der Vorberechnungen durchführt
  \item Optimierung der Anfragen sodass weniger Daten angefragt werden
  \item Laden der Diagramme erst nach Klick darauf
\end{itemize}

Da die erste Option an diesem Punkt zu viel Aufwand und die dritte weder effektiv noch
nutzerfreundlich ist wurde die zweite Option gewählt.\\
Dafür wurde die Klasse \emph{RequestReducer eingeführt}.
Sie stellt Methoden bereit die Anfragen über ein Jahr auf 4 Tage im Monat und Tage auf Stichproben
um 0, 8 und 16 Uhr mit Mittelwerten auf 15 Minuten reduziert. Das hat keinen merklichen Einfluss
auf die Aussagekraft des Diagramms aber reduziert die Datenmenge bis zu 70\% 
(~70 MB -> ~20 MB)
