\section{Testfälle}
\subsection{Basis-Testfälle}
\begin{center}
    \begin{tabular}[h]{|c|l|c|}
        \hline
        \textbf{TF} & \textbf{Bezeichnung} & \textbf{Bestanden} \\
        \hline
        /TF10/ & Webanwendung öffnen & \cellcolor{green!25}OK \\
        \hline
        /TF20/ & Karte bewegen & \cellcolor{green!25}OK \\
        \hline
        /TF30/ & Handylayout & \cellcolor{red!25}FAILED \\
        \hline
        /TF40/ & Zoomen & \cellcolor{green!25}OK \\
        \hline
         /TF50/ & Einen Pin einer Messtation anklicken & \cellcolor{green!25}OK \\
        \hline
        /TF60/ & Karte auswählen & \cellcolor{green!25}OK \\
        \hline
        /TF70/ & Färbungsskala anzeigen & \cellcolor{green!25}OK \\
        \hline
         /TF80/ & Ort suchen & \cellcolor{green!25}OK \\
        \hline
         /TF90/ & Fehlermeldung bei der Suche & \cellcolor{red!25}FAILED \\
        \hline
        /TF100/ & Zum jetzigen Standort springen & \cellcolor{green!25}OK \\
        \hline
        /TF110/ & Scrollen & \cellcolor{green!25}OK \\
        \hline
         /TF120/ & Zeitrahmen einstellen & \cellcolor{green!25}OK \\
        \hline
         /TF130/ & Vergleich mit letztem Jahr & \cellcolor{green!25}OK \\
        \hline
         /TF140/ & Grenzwertüberschreitung & \cellcolor{red!25}FAILED \\
        \hline
        /TF150/ & Zur Karte zurückkehren & \cellcolor{green!25}OK \\
        \hline
         /TF160/ & Webanwendung schließen & \cellcolor{green!25}OK \\
        \hline

    \end{tabular}
\end{center}

\subsection{Erweiterte Testfälle}
\subsection{Stabilitättests}

\subsection{Modifizierte Testfälle}

\begin{center}
   	\item[/TF70/ Färbungsskala anzeigen] \textbf{Änderung:} Kartenansicht ist offen. \\ \textbf{Erwartetes Ergebnis:} Links unten ist eine Färbungsskala zu sehen. Die Skala ändert sich beim Wechseln der Feature, aber nicht beim Zoomen.

\end{center}

\subsection{Gelöschte Testfälle}

\subsection{Erweiterte Testfälle}




