\section{Vorwort}

\subsection{SmartAQnet}
Luftverschmutzung in deutschen Großstädten ist eines der großen Gegenwartsthemen und immer wieder Gegenstand politischer und öffentlicher Diskussion. Die durch hohe Feinstaubwerte hervorgerufenen gesundheitlichen Risiken sind gravierend, weshalb es häufig Maßnahmen bedarf, der Luftverschmutzung entgegenzuwirken.

Um jedoch die komplexen Zusammenhänge verschiedener Einflussfaktoren auf die Luftqualität in Städten besser zu verstehen und damit effektivere Gegenmaßnahmen zu entwickeln, sind hochauflösende Luftqualitätsdaten erforderlich. Dies stellt aber häufig ein Problem dar, da in vielen Städten nur einige, wenige Messstationen existieren.

Um hierfür eine Lösung zu entwickeln, wurde das Projekt SmartAQnet ins Leben gerufen, an dem Wissenschaftler unterschiedlicher Disziplinen und Forschungseinrichtungen mitgewirkt haben und das größtenteils vom Bundesamt für Verkehr und digitale Infrastruktur finanziert wurde. SmartAQnet steht für „Smart Air Quality Network" und hatte zum Ziel ein Gesamtsystem zu schaffen, das die Erfassung, Visualisierung und Vorhersage der räumlichen Verteilung von Luftschadstoffen in städtischen Atmosphären erlaubt.

\subsection{Vermittlung von Luftqualitätsdaten an die interessierte Öffentlichkeit}

Unter anderem wurde im Rahmen von \gls{smartaq} ein heterogenes Sensornetzwerk für Luftqualitätsdaten innerhalb der Modellregion Augsburg geschaffen. Die gemessenen Daten werden auf einer Datenbank gespeichert und stehen über eine API öffentlich zur Verfügung.

Um die Daten jedoch auch Laien besser zugänglich zu machen, soll im Rahmen dieses PSE Projekts eine \gls{Webanwendung} entwickelt werden, die auf die spezifischen Anforderungen interessierter Bürger zugeschnitten ist. Es kommt also vor allem darauf an, die Daten in einer leicht verständlichen, intuitiven Darstellung zu präsentieren, um Nutzern eine maximale Usability und User Experience zu bieten.