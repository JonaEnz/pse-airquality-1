\newglossaryentry{Detailansicht}{
	name=Detailansicht,
	plural=Detailansichten,
	description={Profilansicht einer einzigen \gls{Station}. Hier werden konkrete \glspl{Messwert}, \glspl{Metadatum} und Diagramme
	 und weitere Visualisierungen der \glspl{Messwert} angezeigt. Da nicht jede \gls{Station} die selben \glspl{Feature} misst, 
	 unterscheiden sich die \glspl{Detailansicht} verschiedener \glspl{Station}.}
}

\newglossaryentry{Dropdown-Menu}{
	name=Dropdown-Menü,
	description={Ein Oberflächenelement das verschiedene Optionen zur Auswahl über eine ausklappbare Liste anbietet.}
}


\newglossaryentry{Feature}{
	name=Feature,
	description={Eine bestimmte Eigenschaft der Umgebung die von einer Messstation gemessen wird.}
}

\newglossaryentry{Feinstaubindex}{
	name=Feinstaubindex,
	description={Der normierte (0 µg => 0, Grenzwert => 100) schlechteste Wert der Feinstaubgrößen.}
}

\newglossaryentry{FROST-Server}{
	name=FROST-Server,
	description={Ein Server für SensorThings Daten. \url{https://github.com/FraunhoferIOSB/FROST-Server}}
}

\newglossaryentry{Graph}{
	name=Graph,
	plural=Graphen,
	description={Ein Diagramm das Wertepaare auf zwei Dimensionen mit beschrifteten Achsen abbildet}
}

\newglossaryentry{Handy}{
	name=Smartphone,
	description={Ein internetfähiges Mobiltelefon mit einem Webbrowser der moderne Standards beherrscht.}
}

\newglossaryentry{Kartenansicht}{
	name=Kartenansicht,
	plural=Kartenansichten,
	description={Startansicht der Anwendung. Es wird eine Karte angezeigt, auf der mit Punkten \glspl{Station} eingezeichnet sind.}
}

\newglossaryentry{Kuchendiagramm}{
	name=Kuchendiagramm,
	plural=Kuchendiagramme,
	description={Ein kreisförmiges Diagramm das entlang des Radius in verschiedene Bereiche eingeteilt ist}
}

\newglossaryentry{Live-Karte}{
	name=Live-Karte,
	plural=Live-Karten,
	description={Eine interaktive Karte, die aktuelle \glspl{Messwert} anzeigt. Der Nutzer kann die Karte durch Ziehen mit der Maus seitlich bewegen, sowie Einzoomen und Auszoomen.}
}

\newglossaryentry{Messwert}{
	name=Messwert,
	plural=Messwerte,
	description={Ein Messwert eines Sensors zu einem bestimmten Zeitpunkt}
}

\newglossaryentry{Metadatum}{
	name=Metadatum,
	plural=Metadaten,
	description={Zu einer \gls{Station} gehöriges Datum, welches kein \gls{Messwert} ist. Unter anderem zählen dazu: 
	Geo-Koordinaten einer \gls{Station}, Name einer \gls{Station} und Beschreibung einer \gls{Station}.}
}

\newglossaryentry{Pin}{
	name=Pin,
	plural=Pins,
	description={Ein Oberflächenelement das einen bestimmten Punkt auf einer Karte markiert}
}


\newglossaryentry{Pop-Up}{
	name=Pop-Up,
	plural=Pop-Ups,
	description={Ein Element einer Webseite das sich über andere Elemente legt und durch einen Klick außerhalb des Pop-Ups geschlossen werden.}
}

\newglossaryentry{Querysprache}{
	name=Querysprache,
	description={Sprache in der die Anfragen an den \gls{FROST-Server} formuliert werden.}
}


\newglossaryentry{Seitenmenu}{
	name=Seitenmenü,
	description={Ein Oberflächenelement das sich an der Seite des Fensters befindet und der Navigation dient.}
}

\newglossaryentry{SmartAQnet}{
	name=SmartAQnet,
	description={Projekt Smart Air Quality Network, ein reproduzierbares Messnetzwerk in der Modellregion Augsburg.}
}

\newglossaryentry{Station}{
	name=Station,
	plural=Stationen,
	description={Ein Objekt an einem bestimmten geographischen Ort das verschiedene Messdaten liefert}
}

\newglossaryentry{Tooltip}{
	name=Tooltip,
	plural=Tooltips,
	description={Ein Oberflächenelement das weitere Informationen zu einem Objekt anzeigt wenn die Maus darüber bewegt wird/es angeklickt wird.}
}

\newglossaryentry{Webanwendung}
{
	name=Webanwendung,
	description={Eine Anwendung die in einem Browser ausgeführt und über das Internet übertragen wird.}
}


\newglossaryentry{Webbrowser}{
	name=Webbrowser,
	description={Webbrowser sind spezielle Computerprogramme zur Darstellung von Webseiten im World Wide Web. Beispiele sind}
}



\newglossaryentry{Webserver}{
	name=Webserver,
	description=Ein Server der eine \gls{Webanwendung} bereitstellt.
}