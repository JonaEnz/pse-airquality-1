\section{Zielbestimmung}
Das erklärte Ziel der Anwendung ist es, Bürgern die Messdaten des \gls{SmartAQnet}-Projekts auf eine ansprechende Weise zugänglich zu machen.
Zu diesem Zweck wurde eine Befragung der Zielgruppe durchgeführt, auf deren Basis die angezeigten Informationen ausgewählt wurden.
Es sollen nicht die rohen \glspl{Messwert} im Mittelpunkt stehen sondern die Daten im verständlichen Kontext dargestellt werden.
Auch soll die Bedienung intuitiv ohne Eingewöhnungszeit möglich sein.
Die Anwendung soll die benötigten Daten selbstständig von einem \gls{FROST-Server} anfragen und, außer zur Übertragung der \gls{Webanwendung}, ohne \gls{Webserver} oder eigene Datenbank funktionieren.

\subsection{Musskriterien}
\setcounter{counter}{10}

\subsubsection*{Allgemein}
\begin{Kriterien}{MK}

	\item Angepasste Layouts für Computer und \gls{Handy}
	
	\item Es können REST-Anfragen in der \gls{Querysprache} des \gls{FROST-Server} erzeugt und übertragen werden.
		Dies ist auch asynchron möglich um große Datenmengen nach Laden der Seite nachladen zu können.
	
	\item Aus den Antworten des \gls{FROST-Server} können die benötigten Daten für die \glspl{Graph} und Kartenansichten ausgelesen werden.
	
	\item Aus den \glspl{Messwert}n der verschiedenen Feinstaubgrößen kann ein einzelner \gls{Feinstaubindex} berechnet werden
\end{Kriterien}

\subsubsection*{\gls{Kartenansicht}}
\begin{Kriterien}{MK}

	\item Eine \gls{Live-Karte} zeigt alle \glspl{Station} gleichzeitig. 
	
	\item Der Nutzer kann ein \gls{Feature} auswählen, das auf der \gls{Live-Karte} eingefärbt wird.
	
	\item Der Nutzer kann die Skala der Färbung und eine kurze Erklärung des gewählten \gls{Feature}s anzeigen lassen.
	
	\item Die Nutzer können Straßennamen und Stadtnamen eingeben und die \gls{Live-Karte} springt auf die Stelle. 
		Es gibt eine Fehlermeldung, wenn der Ort nicht existiert.
	
	\item Wenn der Nutzer GPS-Lokalisierung erlaubt, kann der Nutzer per Button die Karte auf seine aktuelle Position zentrieren.

	\item \glspl{Station} haben eine Färbung je nach Bewertung der \glspl{Messwert} auf einer Skala.
	
	\item Beim Klicken auf eine \gls{Station} wird eine \gls{Pop-Up} Box angezeigt mit dem Name der Station, der Position und dem exakten \gls{Messwert}.
		Über eine Schaltfläche kann die \gls{Detailansicht} geöffnet werden. 
\end{Kriterien}

\subsubsection*{\gls{Detailansicht}}
\begin{Kriterien}{MK}
	\item Es werden \glspl{Metadatum} zur gewählten \gls{Station} angezeigt.
	
	\item Es werden die zuletzt gemessenen \glspl{Messwert} der gewählten \gls{Station} angezeigt.

	\item Die zeitliche Veränderung der \glspl{Messwert} an dieser \gls{Station}, wird für die wichtigsten \glspl{Feature} (PM 10, 
	PM 2.5, Temperatur, Luftfeuchtigkeit, Luftdruck) in Form eines Diagramms dargestellt. Der zeitliche Rahmen kann eingestellt 
	werden.

	\item Es werden zusammengefasste Informationen angegeben: 
\end{Kriterien}
		
\begin{itemize}
	\item Einordnung der (historischen) Feinstaubwerte an dieser \gls{Station} im Bezug auf die offiziellen Grenzwerte.
    \item Vergleich mit Durchschnitt der \glspl{Messwert} im Umkreis (20 km).
    \item \gls{Kuchendiagramm}: Anzahl der Tage im gewählten Zeitraum, an denen der Feinstaubwert höher war als heute.
\end{itemize}

\begin{Kriterien}{MK}	

	\item Es ist in einem Klick möglich, von der \gls{Detailansicht} zur \gls{Kartenansicht} zurückzukehren.

\end{Kriterien}

\newpage
\subsection{Wunschkriterien}
\setcounter{counter}{10}
\begin{Kriterien}{WK}

	\item In der \gls{Detailansicht} wird die Position der ausgewählten \gls{Station} auf einer statischen Karte angezeigt 
	(nur für Desktop Layout). 

	\item Zusätzliche Informationen über die Zuverlässigkeit des Sensors werden angezeigt.

	\item Verschiedene Sprachen sind auswählbar (insbesondere Deutsch und Englisch)
	
	\item In der \gls{Kartenansicht} können farbige Polygone über die Karte gelegt werden. Die Ecken der Polygone bilden 
	\glspl{Station}. Die Farbe der Polygone wird aus dem Durchschnitt der \glspl{Messwert} angrenzender \glspl{Station} berechnet.
	
	\item In der \gls{Detailansicht} wird ein \gls{Graph} angezeigt, der die Veränderung des Tagesdurschnitts der Feintstaub 
	\glspl{Feature} über einen einstellbaren Zeitabschnitt (z.B. ein Jahr) anzeigt.
	
	\item Die Webanwendung speichert den Zustand der Karte, wenn der Nutzer eine \gls{Detailansicht} öffnet. Wenn er die 
	\gls{Detailansicht} wieder schließt, wird der vorherige Zustand der Karte wiederhergestellt.

\end{Kriterien}

\subsection{Abgrenzungskriterien}
\setcounter{counter}{10}
\begin{Kriterien}{AK}

	\item Historische Daten werden nicht einzeln, sondern nur in Diagrammen angezeigt.
	
	\item Die Folgen und Ursachen der Feinstaubbelastung werden nicht genannt. Es wird aber auf andere Webseiten verlinkt, die diese
	Informationen bereitstellen.
	
	\item Bei der Nutzung der \gls{Webanwendung} werden keine neue Tabs geöffnet. Alles passiert im gleichen Tab. Außgenommen 
	davon ist das Öffnen von externen Links. Diese Seiten werden in einem neuen Tab geöffnet.
	
	\item Die \gls{Webanwendung} benutzt den \gls{Webserver} nur um die Anwendung zu übertragen. Es werden keine anderen Anfragen 
	an den \gls{Webserver} gestellt.
	
\end{Kriterien}
