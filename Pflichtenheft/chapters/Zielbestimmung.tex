\section{Zielbestimmung}
Das erklärte Ziel der Anwendung ist es, ...

\subsection{Musskriterien}
\setcounter{counter}{10}

\begin{Kriterien}{MK}

	\item Webanwendung in allen Browser die JavaScript unterstützen

	\item Angepasste Layouts für Computer und Handy
	
	\item Eine interaktive Karte zeigt alle Messtationen gleichzeitig, in der Karte kann man ein- und auszoomen, und frei seitlich bewegen.

	\item Messtationen haben eine Färbung, wie hoch die Bewertung der Messwerte auf der skala ist
	
	\item Beim Bewegen des Cursors über eine Messstation, wir ein Pop-up Box angezeigt mit der Straße und mit dem exakten Messwert.
	
	\item Der Nutzer kann auswählen welchen Aspekt der Daten er will, durch einen Kasten mit Buttons. 
	
	\item Für jede Eigenschaft wird ein Diagramm angezeigt, das die Entwicklung der daten zeigt. Die Zeitspanne ist mit einem Schieberegler einstellbar.
	
	\item Ein Diagramm die die Veränderung von dem Durchschnitt anzeigt, Dabei ist der Zeitabschnitt mit einem Schieberegler einstellbar.
	
	\item Es werden auch zusammengefasste Informationen angegeben: 
		\begin{itemize}
		 \item Wie viel Prozent die Messwerte besser/schlechter sind als letztes Jahr.
        \item Vergleich mit Durchschnitt.
        \item Der Feinstaubwert war an xxx Tagen größer als heute(mit PieChart)
		\end{itemize}
		
	\item Ein Button führt zur Karte zurück
 
\end{Kriterien}

\newpage
\subsection{Wunschkriterien}
\setcounter{counter}{10}
\begin{Kriterien}{WK}

	\item Wechseln zwischen verschiedenen interaktiven Karten für die verschiedenen Aspekten der Messdaten. Die Karten werden an einem zusätzlichen Streifen oben ausgewählt.

	\item Auf der Seite mit den Daten einer bestimmten Messtation wird eine Karte gezeigt, wo der Messstation sich genau befindet. 

	\item Wenn bestimmte Daten für bestimmte Messstation nicht existieren und sie aber trotzdem ausgewählt werden, kommt eine Fehlermeldung.

	\item Zusätzliche Informationen über die Zuverlässigkeit des Sensors werden mit einer ausklappbaren Schaltfläche erreichbar.

	\item Für verschiedenen Sprachen Auswahl ermöglichen (insbesondere Deutsch und Englisch)

\end{Kriterien}

\subsection{Abgrenzungskriterien}
\setcounter{counter}{10}
\begin{Kriterien}{AK}

	\item Historische Daten werden nicht einzeln angezeigt, nur die Diagramme geben Informationen.
	
	\item Die Seite gibt keine ausführliche Erklärung dazu, was  genau Messwerte bedeuten, nur die daraus folgernde Luftqualität wird angezeigt. 
	
	\item Die Folgen und Ursachen der Feinstaubbelastung werden nicht genannt.
	
	\item Bei der Nutzung der Webanwendung werden keine neue Tabs geöffnet. Alles passiert im gleichen Tab.
	
\end{Kriterien}
