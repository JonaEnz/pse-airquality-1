\section{Zielbestimmung}
Das erklärte Ziel der Anwendung ist es, Bürgern die Messdaten des \gls{smartaq}-Projekts auf eine ansprechende Weise zugänglich zu machen.
Zu diesem Zweck wurde eine Befragung der Zielgruppe durchgeführt, auf deren Basis die angezeigten Informationen ausgewählt wurden.
Es sollen nicht die rohen \glspl{Messwert} im Mittelpunkt stehen sondern die Daten im verständlichen Kontext dargestellt werden.
Auch soll die Bedienung intuitiv ohne Eingewöhnungszeit möglich sein.
Die Anwendung soll die benötigten Daten selbstständig von einem \gls{FROST-Server} anfragen und, außer zur Übertragung der \gls{Webanwendung}, ohne \gls{Webserver} oder eigene Datenbank funktionieren.

\subsection{Musskriterien}
\setcounter{counter}{10}

\begin{Kriterien}{MK}

	\item \gls{Webanwendung}, lauffähig in allen modernen Browsern die JavaScript unterstützen

	\item Angepasste Layouts für Computer und \gls{Handy}
	
	\item Es können Anfragen in der \gls{Querysprache} des \gls{FROST-Server} erzeugt und übertragen werden.
	
	\item Die Antworten des \gls{FROST-Server} können in die benötigten Formen zur Anzeige umgewandelt werden.
	
	\item Aus den \glspl{Messwert} der verschiedenen Feinstaubgrößen kann ein einzelner \gls{Feinstaubindex} berechnet werden

	\item Eine interaktive Karte zeigt alle \glspl{Station} gleichzeitig, in der Karte kann man ein- und auszoomen, und frei seitlich bewegen.

	\item \glspl{Station} haben eine Färbung je nach Bewertung der \glspl{Messwert} auf der Skala.
	
	\item Beim Bewegen des Cursors über eine \gls{Station} wird eine \gls{Pop-Up} Box angezeigt mit dem Name der Station, der Position und dem exakten Messwert.
		Über eine Schaltfläche kann die Detailansicht geöffnet werden.
	
	\item Der Nutzer kann durch ein \gls{Dropdown-Menu} festlegen welchen Aspekt der Daten angezeigt werden soll. 
	
	\item Für jede Eigenschaft wird ein Diagramm angezeigt, das die Entwicklung der \glspl{Messwert} zeigt. Die Zeitspanne ist einstellbar.
	
	\item Ein \gls{Graph} der die Veränderung von dem Tagesdurchschnitt anzeigt, dabei ist der Zeitabschnitt einstellbar.
	
	\item Es werden auch zusammengefasste Informationen angegeben: 
\end{Kriterien}
		
\begin{itemize}
	\item Wie viel Prozent die \glspl{Messwert} besser/schlechter sind als letztes Jahr.
    \item Vergleich mit Durchschnitt der \glspl{Messwert} im Umkreis.
    \item Der Feinstaubwert war an xxx Tagen größer als heute (mit \gls{Kuchendiagramm})
\end{itemize}

\begin{Kriterien}{MK}	
	\item Ein Button führt zur Karte zurück
 
\end{Kriterien}

\newpage
\subsection{Wunschkriterien}
\setcounter{counter}{10}
\begin{Kriterien}{WK}

	\item Wechseln zwischen verschiedenen interaktiven Karten für die verschiedenen Aspekten der \glspl{Messwert}.

	\item Auf der Seite mit den Daten einer bestimmten \gls{Station} wird auf einer Karte gezeigt, wo die \gls{Station} sich befindet. 

	\item Zusätzliche Informationen über die Zuverlässigkeit des Sensors sind mit einer ausklappbaren Schaltfläche erreichbar.

	\item Verschiedene Sprachen sind auswählbar (insbesondere Deutsch und Englisch)

\end{Kriterien}

\subsection{Abgrenzungskriterien}
\setcounter{counter}{10}
\begin{Kriterien}{AK}

	\item Historische Daten werden nicht einzeln angezeigt, nur die Diagramme geben Informationen.
	
	\item Die Seite gibt keine ausführliche Erklärung dazu, was genau die \glspl{Messwert} bedeuten, nur die daraus folgernde Luftqualität wird angezeigt. 
	
	\item Die Folgen und Ursachen der Feinstaubbelastung werden nicht genannt.
	
	\item Bei der Nutzung der \gls{Webanwendung} werden keine neue Tabs geöffnet. Alles passiert im gleichen Tab.
	
	\item Die \gls{Webanwendung} benutzt den Weberver nur um die Anwendung zu übertragen. 
		Es werden keine anderen Anfragen an den \gls{Webserver} gestellt.
	
\end{Kriterien}
