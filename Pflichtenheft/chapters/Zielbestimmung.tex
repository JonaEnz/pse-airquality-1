\section{Zielbestimmung}
Das erklärte Ziel der Anwendung ist es, Bürgern die Messdaten des \gls{smartaq}-Projekts auf eine ansprechende Weise zugänglich zu machen.
Zu diesem Zweck wurde eine Befragung der Zielgruppe durchgeführt, auf deren Basis die angezeigten Informationen ausgewählt wurden.
Es sollen nicht die rohen \glspl{Messwert} im Mittelpunkt stehen sondern die Daten im verständlichen Kontext dargestellt werden.
Auch soll die Bedienung intuitiv ohne Eingewöhnungszeit möglich sein.
Die Anwendung soll die benötigten Daten selbstständig von einem \gls{FROST-Server} anfragen und, außer zur Übertragung der \gls{Webanwendung}, ohne \gls{Webserver} oder eigene Datenbank funktionieren.

\subsection{Musskriterien}
\setcounter{counter}{10}

\begin{Kriterien}{MK}

	\item \gls{Webanwendung}, lauffähig in allen modernen Browsern die JavaScript unterstützen

	\item Angepasste Layouts für Computer und \gls{Handy}
	
	\item Es können Anfragen in der \gls{Querysprache} des \gls{FROST-Server} erzeugt und übertragen werden.
	
	\item Die Antworten des \gls{FROST-Server} können in die benötigten Formen zur Anzeige umgewandelt werden.
	
	\item Aus den \glspl{Messwert} der verschiedenen Feinstaubgrößen kann ein einzelner \gls{Feinstaubindex} berechnet werden

	\item Eine interaktive Karte zeigt alle \glspl{Station} gleichzeitig. 
	
	\item Der Nutzer kann auswählen, welechen Feature die interaktive Karte anzeigen soll.

	\item \glspl{Station} haben eine Färbung je nach Bewertung der \glspl{Messwert} auf der Skala.
	
	\item Beim Klicken auf eine \gls{Station} wird eine \gls{Pop-Up} Box angezeigt mit dem Name der Station, der Position und dem exakten Messwert.
		Über eine Schaltfläche kann die Detailansicht geöffnet werden. 
	
	\item Für jedes Feature wird ein Diagramm angezeigt, das die Entwicklung der \glspl{Messwert} zeigt. Die Zeitspanne ist einstellbar.
	
	\item Ein \gls{Graph} der die Veränderung von dem Tagesdurchschnitt anzeigt, dabei ist der Zeitabschnitt einstellbar.
	
	\item Es werden auch zusammengefasste Informationen angegeben: 
\end{Kriterien}
		
\begin{itemize}
	\item Wie viel Prozent die \glspl{Messwert} besser/schlechter sind als letztes Jahr.
    \item Vergleich mit Durchschnitt der \glspl{Messwert} im Umkreis.
    \item Der Feinstaubwert war an xxx Tagen größer als heute (mit \gls{Kuchendiagramm})
\end{itemize}

\begin{Kriterien}{MK}	
	\item Zur Karte zurückzukehren ist mit einem Klick möglich.
\end{Kriterien}

\newpage
\subsection{Wunschkriterien}
\setcounter{counter}{10}
\begin{Kriterien}{WK}

	\item Wechseln zwischen verschiedenen interaktiven Karten für die verschiedenen Aspekten der \glspl{Messwert}.

	\item Auf der Seite mit den Daten einer bestimmten \gls{Station} wird auf einer Karte gezeigt, wo die \gls{Station} sich befindet (nur für Desktop Layout). 

	\item Zusätzliche Informationen über die Zuverlässigkeit des Sensors sind erreichbar.

	\item Verschiedene Sprachen sind auswählbar (insbesondere Deutsch und Englisch)
	
	\item Auf der interaktiven Karte werden Städte als farbige Polygonen gekennzeichnet. Das Polygon ist nach dem Durchnitt in der Stadt gefärbt. Mit einem Klick auf das Polygon wird die Karte der Stadt angezeigt, auf der nur die Messtationen markiert sind. 
	
	\item Ein Diagramm  \gls{Graph} zeigt den Verlauf der \glspl{Messwert} Anzeigt und dabei auch den Durchschnitt markiert.
	
	\item Die Nutzer können Straßennamen und Stadtnamen eingeben und die interaktive Karte pringt auf die Stelle.
	
	\item Wenn der Nutzer GPS-Lokalisiernug erlaubt, startet die Karte beim jeden neuen Aufruf der Webanwendung an dem Wohnort des Nutzers.
	
	\item Die Webanwendung speichert, an welcher Stelle die interaktive Karte war, als der Nutzer die Detailansicht geöffnet hat, und öffnet beim Rückkehr wieder an der Stelle. Auch bei dem Wechsel zwischen den Features bleibt die Karte bei der gleichen Stelle und Größe.

\end{Kriterien}

\subsection{Abgrenzungskriterien}
\setcounter{counter}{10}
\begin{Kriterien}{AK}

	\item Historische Daten werden nicht einzeln angezeigt, nur die Diagramme geben Informationen.
	
	\item Die Seite gibt keine ausführliche Erklärung dazu, was genau die \glspl{Messwert} bedeuten, nur die daraus folgernde Luftqualität wird angezeigt. 
	
	\item Die Folgen und Ursachen der Feinstaubbelastung werden nicht genannt.
	
	\item Bei der Nutzung der \gls{Webanwendung} werden keine neue Tabs geöffnet. Alles passiert im gleichen Tab.
	
	\item Die \gls{Webanwendung} benutzt den Weberver nur um die Anwendung zu übertragen. 
		Es werden keine anderen Anfragen an den \gls{Webserver} gestellt.
	
	
\end{Kriterien}
