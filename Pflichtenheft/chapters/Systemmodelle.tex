\section{Systemmodelle}

\subsection{Funktionsdiagramm Frontend}

\subsection{Szenarios}

\subsubsection{Szenario 1: Ansicht, der studenaktuellen Feinstaubwerte (PM 10) am Smartphone}
\textbf{Nutzer:} Nutzer mit grundlegenden IT-Kenntnissen der schon einmal ähnliche Anwendungen bedient hat (z.B. Google Maps) und somit intuitiv die Karte bedienen kann. Des Weiteren hat er jedoch wenig Kenntnisse über Feinstaubwerte und benötigt somit eine Einordnung und Erklärung der \glspl{Messwert}.

\textbf{Beschreibung:} Der Nutzer interessiert sich für die aktuellen Feinstaubwerte in seiner Stadt.

\textbf{Konkreter Ablauf:}
Der Nutzer öffnet die \gls{Webanwendung}, indem er die URL in seinen Smartphone Browser eingibt und bestätigt. Daraufhin öffnet sich die Kartenansicht der Anwendung. Dort sind die \glspl{Station} als Punkt eingetragen. Über einen Button in der rechten unteren Ecke ruft der Nutzer das Konfigurationsmenü der Karte auf. Dort Klickt er auf PM 10. Daraufhin werden nur die \glspl{Station} als Punkt auf der Karte angezeigt, die das \gls{Feature} PM10 messen und sie werden, aufgrund ihres zuletzt gemessenen PM10 Wertes eingefärbt. Desto grüner ein Punkt, desto niedriger der Wert. Desto roter ein Punkt, desto höher ein Wert. Die Farbskala wird in der linke unteren Ecke angezeigt.
Um einen \gls{Messwert} in Zahlenform einer \gls{Station} anzuzeigen, klickt der Nutzer auf einen Punkt. Daraufhin öffnet am unteren Bildschirmrand ein Kasten, indem der Name der \gls{Station}, die Position und der PM10 Wert angezeigt wird. Außerdem wird eine Warnung angezeigt, wenn der \gls{Messwert} einen Grenzwert überschritten hat.

\subsubsection{Szenario 2: Ansicht der Diagramme einer spezifischen Messstation am Computer}
\textbf{Nutzer:} Nutzer mit grundlegenden IT-Kenntnissen der schon einmal ähnliche Anwendungen bedient hat (z.B. Google Maps) und somit intuitiv die Karte bedienen kann. Er hat sich schon häufiger über Luftqualität informiert und ist an konkreten \glspl{Messwert}n und deren zeitlicher Entwicklung interessiert.

\textbf{Beschreibung:} Der Nutzer möchte Diagramme einer \gls{Station} einsehen.

\textbf{Konkreter Ablauf:} Der Nutzer öffnet die \gls{Webanwendung}, indem er die URL in in den Browser seines Computer eintippt und bestätigt. Daraufhin öffnet sich die Kartenansicht der \gls{Webanwendung}. Über das Suchfeld in der linken oberen Ecke gibt er den Namen einer \gls{Station} ein. Daraufhin springt die Karte zum Punkt, der die gesuchte \gls{Station} repressentiert. Desweiteren öffnet sich ein Popup neben dem Punkt. Darin wird der Name der \gls{Station}, ihre Position und ihr letzter \gls{Messwert} des auswewählten Features angezeigt, der in der Kartenkonfiguration eingestellt ist. Desweiteren, wird ein Button angezeigt, auf dem "weitere Informationen" steht. Der Nutzer klickt auf den Button.
Daraufhin wird die Profilansicht der \gls{Station} angezeigt. In der linken oberen Ecke wird der Name und Metainformationen der \gls{Station} angezeigt. In der rechten oberen Ecke wird eine statische Karte angezeigt, auf der die Position der \gls{Station} markiert ist.
Darunter werden Diagramme zu den gemessenen Feature der \gls{Station}. Die angezeigten Diagramme sind von \gls{Station} zu \gls{Station} unterschiedlich, da die \glspl{Station} unterschiedliche Features messen.