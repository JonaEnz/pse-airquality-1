\section{Funktionale Anforderungen}

\begin{itemize}
  \item Die Anforderungen, die zu Wunschkriterien gehören, werden mit * gekennzeichnet.
\end{itemize}

\setcounter{counter}{10}

\subsection{Allgemeine Anforderungen}

\begin{Kriterien}{FA}
  \item[Statische Webanwendung]
    Die \gls{Webanwendung} erfordert den \gls{Webserver} nur um die Anwendung zu übertragen.
    Sonst müssen keine Anfragen an den \gls{Webserver} gestellt werden. 
\end{Kriterien}


\subsection{Kartenansicht}

\begin{Kriterien}{FA}
  \item[Startseite]
    Bei Aufrufen der Webseite wird die \gls{Kartenansicht} angezeigt. Es wird eine frei bewegliche \gls{Live-Karte} 
    angezeigt, in der zunächst alle \glspl{Station} als \glspl{Pin} angezeigt werden.

  \item[Stationen Pop-Up]
    Wird mit der Maus auf einen \gls{Pin} geklickt, öffnet sich ein \gls{Pop-Up}. Darin werden der Name der \gls{Station}, 
    die Koordinaten und der letzte gemessene \gls{Messwert} des ausgewählten \glspl{Feature} angezeigt. Ein Klick auf 
    'Details anzeigen' öffnet die \gls{Detailansicht} der \gls{Station}.
\end{Kriterien}

\subsubsection*{Navigation in der Karte}
\begin{Kriterien}{FA}
  \item[Suche nach Adressen]
    Es gibt ein Suchfeld, über das nach Adressen gesucht werden kann. Existiert die Adresse, so zentriert sich die Karte 
    auf der gesuchten Adresse.

  \item[Suche Fehlermeldung]
    Sucht ein Nutzer nach einer Adresse,  die nicht existiert, so wird eine Fehlermeldung ausgegeben. 

  \item[Standortbestimmung*]
    Es gibt einen Button, mit dem der Nutzer in der Karte zu seinem Standort springen kann. Dazu 
    muss sein Browser Standortbestimmung unterstützen. Klickt der Nutzer auf den Button so zentriert sich die Karte über seinem 
    aktuellen Standort.
  
  \item[Standort Fehlermeldung*]
    Falls kein Standort ermittelt werden kann erscheint eine Fehlermeldung.
\end{Kriterien}

\subsubsection*{Konfiguration und Färbung der Karte}
\begin{Kriterien}{FA}
  \item[Kartenkonfiguration]
    Die \gls{Live-Karte} der \gls{Kartenansicht} lässt sich über die Kartenkonfiguration anpassen. Es kann ausgewählt werden, welches 
    \gls{Feature} eingefärbt werden soll. Wird ein Feature ausgewählt, so werden in der \gls{Live-Karte} nur noch die \glspl{Station} 
    mit einem \gls{Pin} eingezeichnet, die in der letzten Stunde einen \gls{Messwert} des gewählten \glspl{Feature} gemessen 
    haben. Des Weiteren werden die \glspl{Pin} bezüglich dieses Messwertes eingefärbt. Zwischen folgenden \glspl{Feature} 
    kann gewählt werden: Feinstaubindex, PM 10, PM 2.5, Temperatur, Luftfeuchtigkeit und Luftdruck.
    Zuzüglich können spezielle Optionen an- und abgewählt werden, die Darstellung und Färbung der Karte unabhängig vom gewählten 
    Feature verändern.
    
  \item[Färbungsskala und Feature-Info]
    Zu jedem Feature gibt es eine Skala, die einem \gls{Messwert} eine Farbe zuordnet. Nach dieser Skala werden die 
    \glspl{Pin} eingefärbt. Falls vorhanden wird die Skala vom \gls{FROST-Server} abgefragt. Die Skala kann über den Info 
    Button aus- und eingeblendet werden. Zuzüglich zur Skala, wird ein kurzer Infotext zum gewählen Feature angezeigt. Bei 
    Bedarf wird eine externe Webseite verlinkt, in der das Feature noch ausführlicher erklärt wird.

  \item[Option: Flächenwerte*]
    Zwischen den \glspl{Station} werden Dreiecke aufgespannt die entsprechend der Höhe der durchschnittlichen \glspl{Messwert} 
    gefärbt werden. Dabei überschneiden sich die Dreiecke nicht.

  \item[Option: Färbung Umkreisdurchschnitt*]
    Die Färbung der \glspl{Station} wird nun nicht mehr anhand der featurespezifischen Skala vorgenommen, sondern anhand der 
    Durchschnittswerte aller \glspl{Station} im Umkreis von 20 km festgelegt. Je nach dem, wie weit der letzte \gls{Messwert} 
    dieser Station über bzw. unter dem Umkreisdurchschnitt liegt, wird der \gls{Pin} der \gls{Station} unterschiedlich eingefärbt.
    In der Pop-Up-Ansicht einer \gls{Station} wird die prozentuale Über-/Unterschreitung angezeigt.
\end{Kriterien}

\subsubsection*{Sonstiges}
\begin{Kriterien}{FA}
  \item[Überschrittene Grenzwerte*]
    Wurde für ein ausgewähltes \gls{Feature} an einer \gls{Station} am vergangenen Tag ein Grenzwert überschritten, so wird der 
    \gls{Pin} dieser \gls{Station} mit einem '!' markiert. Des Weiteren wird ein Hinweis auf die Überschreitung in der 
    Pop-Up-Ansicht und der \gls{Detailansicht} dieser \gls{Station} angezeigt.

  \item[Letzte Ansicht*]
    Kehrt ein Nutzer zur Seite zurück, wird die \gls{Live-Karte} geladen, die der Nutzer zuletzt verwendet hat. Es wird also 
    gespeichert, welches Feature der Nutzer zuletzt ausgewählt hat und worauf er die Karte (per Adresssuche oder per 
    Standortbestimmung) zuletzt zentriert hat.
\end{Kriterien}


\subsection{Detailansicht}

\begin{Kriterien}{FA}
  \item[Detailansicht]
    In der Detailansicht werden verschiedene \glspl{Graph} und \glspl{Messwert} angezeigt. 
    (Siehe "Diagramme, Graphen und Werte")

  \item[Positionsanzeige]
    Neben dem Name der \gls{Station} befindet sich ein Kartenausschnitt der die Position der \gls{Station} mit einem \gls{Pin} 
    markiert. Mit dieser Karte kann nicht interagiert werden.

  \item[Warnung bei Grenzwertüberschreitung]
     Eine Warnung wird in der Detailansicht neben einem \gls{Messwert} angezeigt, falls dieser einen Grenzwert überschritten 
     hat. (Im Tagesdurchschnitt des letzten Tages)

  \item[Dynamische Anpassung nach Sensor]
    Da die \glspl{Station} verschiedene \glspl{Feature} bereitstellen können passt sich die Detailansicht entsprechend an. Es 
    werden immer nur \glspl{Graph} mit verfügbaren \glspl{Messwert} angezeigt. Welche \glspl{Graph} für ein \gls{Feature} 
    angezeigt werden ist für jedes \gls{Feature} festgelegt.
\end{Kriterien}

 \subsubsection*{Diagramme, Graphen und Werte}
 \begin{Kriterien}{FA}
  \item[Historische Entwicklung]
    Ein \gls{Graph} der die Entwicklung der \glspl{Messwert} darstellt wird anzeigt. Der Zeitrahmen kann in vorgegebenen Stufen 
    eingestellt werden.

  \item[Jahresvergleich*]
    Ein Diagramm, das den Jahresverlauf der \glspl{Messwert} verschiedener Jahre zeigt wird dargestellt.
    Die X-Achse umfasst Januar - Dezember, mehrere \glspl{Graph} zeigen die Jahresverläufe an.

  \item[Heute im Vergleich zum letzten Jahr*]
    Ein \gls{Kuchendiagramm}, das die Anzahl der Tage im letzten Jahr visualisiert, an denen die \glspl{Messwert} höher/niedriger 
    waren als heute.

  \item[Weitere Informationen*]
    Wird bei \glspl{Graph} des \gls{Feinstaubindex} (i)-Symbol angeklickt öffnet sich ein \gls{Pop-Up} mit einer Erklärung für den 
    \gls{Feinstaubindex}.
    Diese Erklärung erläutert wie genau der \gls{Feinstaubindex} aus den \glspl{Messwert} bestimmt wird.
\end{Kriterien}


\subsection{Weitere Anforderungen*}

\begin{Kriterien}{FA}
  \item[Ladeanzeige]
    Während im Hintergrund Daten vom \gls{FROST-Server} angefragt werden wird ein Ladesymbol angezeigt.

  \item[Server nicht erreichbar]
    Ist der \gls{FROST-Server} nicht erreichbar wird eine Fehlermeldung angezeigt und die Möglichkeit angeboten es erneut zu 
    versuchen.

  \item[Fehlerseite]
    Wird eine Seite der \gls{Webanwendung} aufgerufen die nicht verfügbar ist wird eine 404-Fehlerseite angezeigt.

  \item[Hamburgermenü]
    Auf jeder Seite gibt es ein \gls{Seitenmenu}, das über den Hamburgerbutton $\equiv$ geöffnet und geschlossen werden kann.
    Hier befinden sich unten aufgeführte Einstellungen und Navigationselemente.

  \item[Sprachauswahl]
    Über das \gls{Seitenmenu} kann eine Sprache gewählt werden. Der Text aller Seiten wird daraufhin in der gewählten Sprache 
    angezeigt.

  \item[Datenschutzerklärung]
    Es gibt eine Seite mit einer Datenschutzerklärung, die im \gls{Seitenmenu} verlinkt ist.

  \item[Impressum]
    Es gibt ein Impressum, das im \gls{Seitenmenu} verlinkt ist.
\end{Kriterien}
