\section{Funktionale Anforderungen}

\begin{itemize}
  \item Die Anforderungen, die zu Wunschkriterien gehören, werden mit * gekennzeichnet.
\end{itemize}

\setcounter{counter}{10}
\begin{Kriterien}{FA}

\subsection{Allgemeine Anforderungen}

 \item[Statische Webanwendung]
   Die \gls{Webanwendung} erfordert den \gls{Webserver} nur um die Anwendung zu übertragen.
   Sonst müssen keine Anfragen an den \gls{Webserver} gestellt werden. 

\subsection{Karte}

 \item[Startseite]
   Bei Aufrufen der Webseite wird die Startseite angezeigt.
   Es wird eine frei bewegliche \gls{Live-Karte} mit allen \glspl{Station} als \glspl{Pin} angezeigt und es kann nach einer Adresse gesucht werden.
   Ein Klick auf 'Ansicht' öffnet ein \gls{Pop-Up} mit \glspl{Feature}.

 \item[Stationen Pop-Up]
  Wird mit der Maus über einen Stations-\gls{Pin} gefahren/mit dem Finger darauf geklickt öffnet sich ein \gls{Pop-Up}.
  Darin werden der Name der \gls{Station}, die Koordinaten und den aktuellen exakten \gls{Messwert}n der \glspl{Feature} angezeigt.
  Ein Klick auf 'Details anzeigen' öffnet die \gls{Detailansicht}.

 \item[Features]
   Die \gls{Live-Karte} auf der Startseite lässt sich über \glspl{Feature} anpassen.
   Es kann zwischen Feinstaub, Temperatur, Luftfeuchtigkeit, Luftdruck, Windgeschwindigkeit und Niederschlagsmenge ausgewählt werden welche \glspl{Messwert} angezeigt werden.
   Es werden nur \glspl{Station} angezeigt die entsprechende Daten besitzen.
   Verschiedene Optionen (siehe "Kartenansichten") zur Einfärbung der Karte und \glspl{Station} sind verfügbar.

 \item[Legende]
  Auf der \gls{Live-Karte} wird eine Legende angezeigt welche die aktuellen Einfärbungen erklärt.
  Dies kann beispielsweise durch eine Farbskala erfolgen.

 \item[Letzte Ansicht*]
  Kehrt ein Nutzer zur Seite zurück wird die \gls{Live-Karte} von der zuletzt genutzten \gls{Feature} geladen.

 \subsubsection*{Kartenansichten}

 \item[Standard]
  \glspl{Station} werden angezeigt und nach \gls{Messwert} eingefärbt. 
  Die Farbskala dafür wird, wenn vorhanden, vom \gls{FROST-Server} angefragt.
  Überschrittene Grenzwerte werden zudem mit einem '!' markiert.

 \item[Flächenwerte*]
   Zwischen den \glspl{Station} werden Dreiecke aufgespannt die entsprechend der Höhe der durchschnittlichen \glspl{Messwert} gefärbt werden.
   Dabei überschneiden sich die Dreiecke nicht.

 \item[Veränderung]
   Die \gls{Station} wird entsprechend der Veränderung über den letzten Monat/Jahr eingefärbt.
   Die \glspl{Tooltip} zeigen die prozentuale Veränderung über diesen Zeitraum an.

 \item[Umkreisdurchschnitt]
  Die Färbung der \glspl{Station} wird anhand des Durchschnitts aller \glspl{Station} im Umkreis von 20 km festgelegt.
  Die \glspl{Tooltip} zeigen den Wert als prozentuale Über-/Unterschreitung dieses Durchschnitts an. 

 \item[Adresse]
   Es kann eine Adresse eingegeben werden die auf der Karte eingezeichnet wird.
   Sie wird wie eine \gls{Station} eingefärbt anhand eines durch benachbarte \glspl{Station} abgeschätzten \gls{Messwert}s.
   Wenn die Adresse nicht existiert, erscheint ein \gls{Pop-Up} mit einer Fehlermeldung.
   
   \item[Jetziger Standort*] Falls der Nutzer Standortermittlung an hat, wird der Standort ermittelt. 
    Drückt der Nutzer den Button 'Zum jetzigen Standort springen', springt die Karte zu dem ermittelten Standort. 
    Falls kein Standort ermittelt werden kann erscheint eine Fehlermeldung.

\subsection{Detailansicht}

 \item[Detailansicht]
   In der Detailansicht werden verschiedene \glspl{Graph} und \glspl{Messwert} angezeigt. 
   (Siehe "Diagramme, Graphen und Werte")
 \item[Positionsanzeige]
  Neben dem Name der \gls{Station} befindet sich ein Kartenausschnitt der die Position der \gls{Station} mit einem \gls{Pin} markiert.
  Mit dieser Karte kann nicht interagiert werden.

 \item[Dynamische Anpassung nach Sensor]
   Da die \glspl{Station} verschiedene \glspl{Feature} bereitstellen können passt sich die Detailansicht entsprechend an.
   Es werden immer nur \glspl{Graph} mit verfügbaren \glspl{Messwert} angezeigt.
   Welche \glspl{Graph} für ein \gls{Feature} angezeigt werden ist für jedes \gls{Feature} festgelegt.

 \item[Warnung bei Grenzwertüberschreitung]
  Eine Warnung wird in der Detailansicht neben einem \gls{Messwert} angezeigt, falls dieser einen Grenzwert überschreitet

 \subsubsection*{Diagramme, Graphen und Werte}

 \item[Historische Entwicklung]
   Ein \gls{Graph} der die Entwicklung der \glspl{Messwert} darstellt wird anzeigt.
   Der Zeitrahmen kann in vorgegebenen Stufen eingestellt werden.
 
 \item[Veränderung Durchschnitt]
   Ein Graph der die Veränderung der \glspl{Messwert} darstellt wird anzeigt.
   Der Zeitrahmen kann in vorgegebenen Stufen eingestellt werden.

 \item[Jahresvergleich*]
   Ein Graph der den Jahresverlauf der \glspl{Messwert} mit anderen Jahren vergleicht wird angezeigt.
   Die X-Achse umfasst Januar - Dezember, mehrere \glspl{Graph} zeigen die Jahresverläufe an.

 \item[Heute im Vergleich zum letzten Jahr]
   Ein \gls{Kuchendiagramm} das die Anzahl der Tage im letzten Jahr visualisiert an denen die \glspl{Messwert} höher/niedriger waren als heute.

 \item[Weitere Informationen*]
   Wird bei \glspl{Graph} des \gls{Feinstaubindex} (i)-Symbol angeklickt öffnet sich ein \gls{Pop-Up} mit einer Erklärung für den \gls{Feinstaubindex}.
   Diese Erklärung erläutert wie genau der \gls{Feinstaubindex} aus den \glspl{Messwert} bestimmt werden.

\subsection{Weitere Anforderungen*}

 \item[Hamburgermenü]
  Auf jeder Seite gibt es ein \gls{Seitenmenu} das über den Hamburgerbutton $\equiv$ geöffnet und geschlossen werden kann.
  Hier befinden sich verschiedene Einstellungen (/FA240/) und Navigationselemente (/FA280/).

 \item[Sprachauswahl]
   Auf jeder Seite befindet sich eine Flagge im \gls{Seitenmenu} welche die aktuell ausgewählte Sprache repräsentiert.
   Ein \gls{Dropdown-Menu} wechselt zwischen den verfügbaren Sprachen.

 \item[Ladeanzeige]
  Während im Hintergrund Daten vom \gls{FROST-Server} angefragt werden wird ein Ladesymbol angezeigt.
  Läuft mehr als eine Anfrage gleichzeitig wird ein Fortschrittsbalken angezeigt.

 \item[Server nicht erreichbar]
  Ist der \gls{FROST-Server} nicht erreichbar wird eine Fehlermeldung angezeigt und die Möglichkeit angeboten es erneut zu versuchen.

 \item[Fehlerseite]
  Wird eine Seite der \gls{Webanwendung} aufgerufen die nicht verfügbar ist wird eine 404-Fehlerseite angezeigt.

 \item[Datenschutzerklärung]
  Im \gls{Seitenmenu} gibt es einen Link zur Datenschutzerklärung.
  Die Datenschutzerklärung erscheint als \gls{Pop-Up}.
\end{Kriterien}
