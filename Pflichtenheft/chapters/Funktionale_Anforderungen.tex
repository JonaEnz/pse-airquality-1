\section{Funktionale Anforderungen}

\setcounter{counter}{10}
\begin{Kriterien}{FA}

\subsection{Allgemeine Anforderungen}

 \item[Unterstützung mobile Endgeräte]
   Die \gls{Webanwendung} ist ohne jegliche Einschränkung auf mobilen Endgeräten nutzbar.  

 \item[Statische Webanwendung]
   Die \gls{Webanwendung} erfordert den \gls{Webserver} nur um die Anwendung zu übertragen.
   Sonst müssen keine Anfragen an den \gls{Webserver} gestellt werden. 

\subsection{Karte}

 \item[Startseite]
   Bei Aufrufen der Webseite wird die Startseite angezeigt.
   Es wird eine frei bewegliche Karte mit allen \glspl{Station} als \glspl{Pin} angezeigt und es kann nach einer Adresse gesucht werden.
   Ein Klick auf 'Ansicht' öffnet ein \gls{Pop-Up} mit Ansichten.

 \item[Stationen Pop-Up]
  Wird mit der Maus über einen Stations-\gls{Pin} gefahren/mit dem Finger darauf geklickt öffnet sich ein \gls{Pop-Up}.
  Darin werden der Name der \gls{Station}, die Koordinaten und den aktuellen exakten Messwerten angezeigt.
  Ein Klick auf 'Details anzeigen' öffnet die Detailansicht.

 \item[Ansichten]
   Die Karte auf der Startseite lässt sich über Ansichten anpassen.
   Es kann ausgewählt werden welche \glspl{Messwert} angezeigt werden.
   Es werden nur \glspl{Station} angezeigt die entsprechende Daten besitzen.
   Verschiedene Optionen zur Einfärbung der Karte und \glspl{Station} sind verfügbar.

 \item[Legende]
  Auf der Karte wird eine Legende angezeigt welche die aktuellen Einfärbungen erklärt  

 \item[Letzte Ansicht]
  Kehrt ein Nutzer zur Seite zurück wird die zuletzt genutzte Ansicht geladen.

 \subsubsection{Kartenansichten}

 \item[Standard]
  \glspl{Station} werden angezeigt und nach \glspl{Messwert} eingefärbt. 
  Überschrittene Grenzwerte werden zudem mit einem '!' markiert.

 \item[Flächenwerte (WK)]
   Zwischen den \glspl{Station} werden Dreiecke aufgespannt die entsprechend der Höhe der durchschnittlichen \glspl{Messwert} gefärbt werden.
   Dabei überschneiden sich die Dreiecke nicht.

 \item[Veränderung]
   Die \gls{Station} wird entsprechend der Veränderung über den letzten Monat/Jahr eingefärbt.
   Die \glspl{Tooltip} zeigen die prozentuale Veränderung an.

 \item[Stadtdurchschnitt]
  Die Färbung der \glspl{Station} wird anhand des Durchschnitts aller \glspl{Station} der Stadt festgelegt.
  Die \glspl{Tooltip} zeigen den Wert als prozentuale Über-/Unterschreitung dieses Durschschnitts an. 

 \item[Adresse]
   Es kann eine Adresse eingegeben werden die auf der Karte eingezeichnet wird.
   Sie wird wie eine \gls{Station} eingefärbt anhand eines durch benachbarte \glspl{Station} abgeschätzten \glspl{Messwert}.

\subsection{Detailansicht}

 \item[Detailansicht]
   In der Detailansicht werden verschiedene \glspl{Graph} und \glspl{Messwert} angezeigt. 

 \item[Dynamische Anpassung nach Sensor]
   Da die \glspl{Station} verschiedene \glspl{Feature} bereitstellen können passt sich die Detailansicht entsprechend an.
   Es werden immer nur \glspl{Graph} mit verfügbaren \glspl{Messwert} angezeigt.
   Welche \glspl{Graph} für ein \gls{Feature} angezeigt werden ist für jedes \gls{Feature} festgelegt.

 \item[Warnung bei Grenzwertüberschreitung]
  Eine Warnung wird in der Detailansicht neben einem \gls{Messwert} angezeigt, falls dieser einen Grenzwert überschreitet

 \subsubsection{Diagramme, Graphen und Werte}

 \item[Historische Entwicklung]
   Ein Graph der die Entwicklung der \glspl{Messwert} darstellt wird anzeigt.
   Der Zeitrahmen kann in vorgegebenen Stufen eingestellt werden.
 
 \item[Veränderung Durchschnitt]
   Ein Graph der die Veränderung der \glspl{Messwert} darstellt wird anzeigt.
   Der Zeitrahmen kann in vorgegebenen Stufen eingestellt werden.

 \item[Jahresvergleich (WK)]
   Ein Graph der den Jahresverlauf der \glspl{Messwert} mit anderen Jahren vergleicht wird angezeigt.
   Die X-Achse umfasst Januar - Dezember, mehrere Graphen zeigen die Jahresverläufe an.

 \item[Heute im Vergleich zum letzten Jahr]
   Ein \gls{Kuchendiagramm} das die Anzahl der Tage im letzten Jahr visualisiert an denen die \glspl{Messwert} geringfügig/sehr höher/niedriger waren als heute.

 \item[Weitere Informationen (WK)]
   Wird bei \glspl{Graph} des \gls{Feinstaubindex} (i)-Symbol angeklickt öffnet sich ein \gls{Pop-Up} mit einer Erklärung für den \gls{Feinstaubindex}.
   Diese Erklärung erläutert wie genau der \gls{Feinstaubindex} aus den \glspl{Messwert} bestimmt werden.

\subsection{Weitere Anforderungen}

 \item[Hamburgermenü]
  Auf jeder Seite gibt es ein \gls{Seitenmenu} das über den Hamburgerbutton $\equiv$ geöffnet und geschlossen werden kann.
  Hier befinden sich verschiedene Einstellungen und Navigationselemente.

 \item[Sprachauswahl]
   Auf jeder Seite befindet sich eine Flagge im \gls{Seitenmenu} welche die aktuell ausgewählte Sprache repräsentiert.
   Ein \gls{Dropdown-Menu} wechselt zwischen den verfügbaren Sprachen.

 \item[Ladeanzeige]
  Während im Hintergrund Daten vom \gls{FROST-Server} angefragt werden wird ein Ladesymbol angezeigt.
  Läuft mehr als eine Anfrage gleichzeitig wird ein Fortschrittsbalken angezeigt.

 \item[Server nicht erreichbar]
  Ist der \gls{FROST-Server} nicht erreichbar wird eine Fehlermeldung angezeigt und die Möglichkeit angeboten es erneut zu versuchen.

 \item[Fehlerseite]
  Wird eine Seite der \gls{Webanwendung} aufgerufen die nicht verfügbar ist wird eine 404-Fehlerseite angezeigt.

 \item[Datenschutzerklärung]
  Im \gls{Seitenmenu} gibt es einen Link zur Datenschutzerklärung.
  Die Datenschutzerklärung erscheint als \gls{Pop-Up}.
\end{Kriterien}