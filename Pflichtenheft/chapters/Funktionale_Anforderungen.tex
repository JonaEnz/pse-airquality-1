\section{Funktionale Anforderungen}

\begin{Kriterien}{FA}

 \item[Unterstützung Mobile Endgeräte]
   Die Webanwendung ist ohne jegliche Einschränkung auf mobilen Endgeräten nutzbar.  

 \item[Statische Webanwendung]
   Die Webanwendung erfordert den Server nur um die Anwendung zu übertragen.
   Sonst müssen keine Anfragen an den Server gestellt werden. (TODO: Eigentlich Abgrenzungkriterium)

 \item[Startseite]
   Beim Aufrufen von \"/\" wird die Startseite angezeigt.
   Es wird eine Karte mit allen \glspl{Station} als \glspl{Pin} angezeigt und es kann nach einer Adresse gesucht werden.
   Bei Klick auf eine Station wird die Detailansicht angezeigt.
   Ein Klick auf 'Filtern' öffnet ein \gls{Pop-Up} mit Filteroptionen.

 \item[Filterung Karte]
   Die Karte auf der Startseite lässt sich über Filter anpassen.
   Es kann ausgewählt werden welche \glspl{Messwert} angezeigt werden.
   Es werden nur \glspl{Station} angezeigt die entsprechende Daten besitzen.
   Verschiedene Optionen zur Einfärbung der Karte und \glspl{Station} sind verfügbar.

 \item[Detailansicht]
   In der Detailansicht werden verschiedene \glspl{Graph} und \glspl{Messwert} angezeigt.
   Es kann in die Konfigurationsansicht gewechselt werden.

 \item[Konfiguration]
   Es kann frei eingestellt werden welche Graphen und \glspl{Messwert} in der Detailansicht angezeigt werden.
   Nach jeder Änderung wird automatisch gespeichert. 

 \item[Historische Entwicklung]
   Ein Graph der die Entwicklung der \glspl{Messwert} dastellt wird anzeigt.
   Der Zeitrahmen kann mit einem Schieberegler eingestellt werden.
 
 \item[Veränderung Durchschnitt]
   Ein Graph der die Veränderung der \glspl{Messwert} dastellt wird anzeigt.
   Der Zeitrahmen kann mit einem Schieberegler eingestellt werden.

 \item[Jahresvergleich]
   Ein Graph der den Jahresverlauf der \glspl{Messwert} mit anderen Jahren vergleicht wird angezeigt.
   Die X-Achse umfasst Januar - Dezember, mehrere Graphen zeigen die Jahrenverläufe an.

 \item[Heute im Vergleich zum letzten Jahr]
   Ein \gls{Kuchendiagramm} das die Anzahl der Tage im letzten Jahr visualisiert an denen die Messwere gerimgfügig/sehr höher/niedriger waren als heute.

 \item[Weitere Informationen]
   Wird bei \glspl{Graph} das (i)-Symbol angeklickt öffnet sich ein \gls{Pop-Up} mit einer Erklärung für die angezeigten Werte.
   Diese Erklärung erläutert wie genau die vorliegenden Werte aus den \glspl{Messwert} bestimmt werden.

 \item[Sprachauswahl]
   Auf jeder Seite befindet sich eine Flagge welche die aktuell ausgewählte Sprache repräsentiert.
   Ein Klick darauf wechselt zwischen Deutsch und Englisch.

 \item[Dynamische Anpassung nach Sensor]
   Da die \glspl{Station} verscheidene \glspl{Messwert} bereitstellen können passt sich die Detailansicht entsprechend an.
   Es werden immer nur \glspl{Graph} mit verfügbaren \glspl{Messwert} angezeigt.
\end{Kriterien}