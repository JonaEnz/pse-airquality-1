\section{Funktionale Anforderungen}

\setcounter{counter}{10}
\begin{Kriterien}{FA}

\subsection{Allgemeine Anforderungen}

 \item[Unterstützung Mobile Endgeräte]
   Die Webanwendung ist ohne jegliche Einschränkung auf mobilen Endgeräten nutzbar.  

 \item[Statische Webanwendung]
   Die Webanwendung erfordert den Server nur um die Anwendung zu übertragen.
   Sonst müssen keine Anfragen an den Server gestellt werden. (TODO: Eigentlich Abgrenzungkriterium)

\subsection{Karte}

 \item[Startseite]
   Bei Aufrufen der Webseite wird die Startseite angezeigt.
   Es wird eine Karte mit allen \glspl{Station} als \glspl{Pin} angezeigt und es kann nach einer Adresse gesucht werden.
   Bei Klick auf eine Station wird die Detailansicht angezeigt.
   Ein Klick auf 'Filtern' öffnet ein \gls{Pop-Up} mit Filteroptionen.

 \item[Filterung Karte]
   Die Karte auf der Startseite lässt sich über Filter anpassen.
   Es kann ausgewählt werden welche \glspl{Messwert} angezeigt werden.
   Es werden nur \glspl{Station} angezeigt die entsprechende Daten besitzen.
   Verschiedene Optionen zur Einfärbung der Karte und \glspl{Station} sind verfügbar.

 \subsubsection{Filteroptionen}

 \item[Flächenwerte]
   Zwischen den \glspl{Station} werden Dreiecke aufgespannt die entsprechend der Höhe der Durchschnittswerte gefärbt werden.
   Dabei überschneiden sich die Dreiecke nicht.

 \item[Veränderung]
   Die \gls{Station} wird entsprechend der Veränderung über den letzten Monat/Jahr eingefärbt.
   Die \glspl{Tooltip} zeigen die prozentuale Veränderung an.

 \item[Stadtdurchschnitt]
  Die Färbung der \glspl{Station} wird anhand des Durchschnitts aller \glspl{Station} der Stadt festgelegt.
  Die \glspl{Tooltip} zeigen den Wert als prozentuale Über-/Unterschreitung dieses Durschschnitts an. 

 \item[Adresse]
   Es kann eine Adresse eingegeben werden die auf der Karte eingezeichnet wird.
   Sie wird wie eine \gls{Station} eingefärbt anhand eines durch benachbarte \glspl{Station} abgeschätzten \glspl{Messwert}.

\subsection{Detailansicht}

 \item[Detailansicht]
   In der Detailansicht werden verschiedene \glspl{Graph} und \glspl{Messwert} angezeigt.
   Es kann in die Konfigurationsansicht gewechselt werden.

 \item[Konfiguration]
   Es kann frei eingestellt werden welche Graphen und \glspl{Messwert} in der Detailansicht angezeigt werden.
   Nach jeder Änderung wird automatisch gespeichert. 

 \item[Dynamische Anpassung nach Sensor]
   Da die \glspl{Station} verscheidene \glspl{Messwert} bereitstellen können passt sich die Detailansicht entsprechend an.
   Es werden immer nur \glspl{Graph} mit verfügbaren \glspl{Messwert} angezeigt.

 \item[Messdatenwechsel]
  Über eine Auswahlbox kann der genutzte \gls{Messwert} für die \glspl{Graph} gewechselt werden.
  Ist der aktuell ausgewählte \gls{Messwert} nicht für diese \gls{Station} verfügbar wird zu einem verfügbarem \gls{Messwert} gewechselt.

 \subsubsection{Diagramme, Graphen und Werte}

 \item[Historische Entwicklung]
   Ein Graph der die Entwicklung der \glspl{Messwert} dastellt wird anzeigt.
   Der Zeitrahmen kann mit einem Schieberegler eingestellt werden.
 
 \item[Veränderung Durchschnitt]
   Ein Graph der die Veränderung der \glspl{Messwert} dastellt wird anzeigt.
   Der Zeitrahmen kann mit einem Schieberegler eingestellt werden.

 \item[Jahresvergleich]
   Ein Graph der den Jahresverlauf der \glspl{Messwert} mit anderen Jahren vergleicht wird angezeigt.
   Die X-Achse umfasst Januar - Dezember, mehrere Graphen zeigen die Jahrenverläufe an.

 \item[Heute im Vergleich zum letzten Jahr]
   Ein \gls{Kuchendiagramm} das die Anzahl der Tage im letzten Jahr visualisiert an denen die Messwere gerimgfügig/sehr höher/niedriger waren als heute.

 \item[Weitere Informationen]
   Wird bei \glspl{Graph} das (i)-Symbol angeklickt öffnet sich ein \gls{Pop-Up} mit einer Erklärung für die angezeigten Werte.
   Diese Erklärung erläutert wie genau die vorliegenden Werte aus den \glspl{Messwert} bestimmt werden.

\subsection{Weitere Anforderungen}

 \item[Sprachauswahl]
   Auf jeder Seite befindet sich eine Flagge welche die aktuell ausgewählte Sprache repräsentiert.
   Ein Klick darauf wechselt zwischen Deutsch und Englisch.

 \item[Ladeanzeige]
  Während im Hintergrund Daten vom \gls{FROST-Server} angefragt werden wird ein Ladesymbol angezeigt.
  Läuft mehr als eine Anfrage gleichzeitig wird ein Fortschrittsbaalken angezeigt.

 \item[Server nicht erreichbar]
  Ist der \gls{FROST-Server} nicht erreichbar wird eine Fehlermeldung angezeigt und die Möglichkeit angeboten es erneut zu versuchen.

 \item[Datenschutzerklärung]
  Auf jeder Seite gibt es einen Link zur Datenschutzerklärung.

    \item Als Startseite wird dem Benutzer eine Karte angezeigt
    \item Messtationen werden durch einen Punkt auf der Karte markiert
    \item Die Punkte der Messtationen können in Bezug auf die aktuellen Messdaten eingefärbt werden
    \item Durch ein Klick auf einen Punkt kann die Detailansicht der Messtationen aufgerufen werden
    \item In der Detailansicht einer Messtationen werden alle aktuellen Messwerte angezeigt
    \item Eine Warnung wird in der Detailansicht neben einem Messwert angezeigt, falls dieser einen Grenzwert überschreitet
    \item Über eine Suchfeld können Messtationen mit dem Namen gesucht werden
    \item 
\end{Kriterien}