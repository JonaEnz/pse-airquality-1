\section{Nichtfunktionale Anforderungen}

\setcounter{counter}{10}
\begin{Kriterien}{NA}
    \item[Verlässlichkeit]
        Die \gls{Webanwendung} soll mindestens 99,99\% der Zeit verfügbar sein.
        Dabei gilt Zeit zu der der \gls{FROST-Server} Fehler aufweist als verfügbar solange eine Fehlermeldung korrekt angezeigt wird.
    \item[Stationsanzahl]
        Es können sich bis zu 1.000 Messstationen auf dem \gls{FROST-Server} befinden bevor Performanceeinbußen bemerkbar werden.
    \item[Nutzerzahl]
        Es können so viele Nutzer auf die \gls{Webanwendung} zugreifen wie der \gls{Webserver} und \gls{FROST-Server} erlauben.
        Die \gls{Webanwendung} selbst läuft lokal und ist daher auf unbegrenzt vielen Endgeräten gleichzeitig ausführbar.
    \item[Usability]
        Die Bedienung der \gls{Webanwendung} ist auch für ungeübte Nutzer intuitiv möglich .
        %und unterstützt sie mit Bedienhilfen (Wunschkriterium).
        %Zu Bedienhilfen zählen beispielsweise \glspl{Tooltip}, die beim ersten Besuch durch die Anwendung leiten.
    \item[Seitenaufbau]
        Der Seitenaufbau sollte zuzüglich der Zeit zur Datenübertragung innerhalb von fünf Sekunden erfolgen.
\end{Kriterien}