\section{Globale Testfälle}
\subsection{Testfälle}
\setcounter{counter}{10}
<<<<<<< HEAD
\begin{itemize}
	\item Für jede einzelne Funktion, die das Programm ausführen soll, muss es einen Testfall geben. Folgendes sind die Testfälle für diese Webanwendung:
\end{itemize}
=======

Für jede einzelne Funktion, die das Programm ausführen soll, muss es einen Testfall geben. Folgendes sind die Testfälle für diese 

Webanwendung:
>>>>>>> f9adc6ec8dbbf1ce2243ac4635b01249cf0a5f26
\subsubsection{Basis-Testfälle}
\begin{Kriterien}{MK}

	\item[T010] Seite aufrufen

	\item[T020] Karte bewegen
	
	
	\item[T030] Rein-/Rauszoomen
	
	\item[T040] Kursor bewegen
	
	\item[T050] Einen Punkt anklicken
	
	\item[T060] Einen Aspekt aus der Liste auswählen
	
	\item[T070] Scrollen
	
	\item[T080] Schieberegler einstellen
	
	\item[T090] Zur Karte zurückkehren
	
	\item[T100] Seite schließen
\end{Kriterien}
\subsubsection{Erweiterte Testfälle}
\begin{Kriterien}{MK}

	\item[T110] Aufklapp-Button aufklappen

	\item[T120]  Aufklapp-Button schließen
	
	\item[T130] Zwischen verschiedenen Karten wechseln
	
\end{Kriterien}
\subsubsection{Stabilitätstests}
\begin{Kriterien}{MK}

	\item[T140] Viele datn gleichzeitig anfordern

	\item[T150] Schnelles Anforddern von daten
	
\end{Kriterien}
