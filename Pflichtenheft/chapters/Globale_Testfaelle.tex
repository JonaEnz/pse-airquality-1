\section{Globale Testfälle}
\subsection{Testfälle}
\setcounter{counter}{10}
\subsubsection{Basis-Testfälle}
\begin{Kriterien}{MK}

	\item[Webanwendung öffnen] Der Benutzer startet die \gls{Webanwendung}. Erwartetes Ergebnis: Startseite mit einer Kartenansicht öffnet sich.

	\item[Karte bewegen] Der Benutzer zieht mit dem Eingabegerät an der Karte. Erwartetes Ergebnis: Die Karte bewegt sich.
	
	\item[Zoomen] Der Benutzer führt mit den Fingern eine Zoom-Geste aus oder Scrollt mit dem Mausrad: Die Karte zoomt.
	
	\item[Einen Pin einer Messtation anklicken] Der Benutzer klickt einen Pin einer Messtation auf der Karte an. Erwartetes Ergebnis: Das Stationen Pop-Up erscheint.
	
	\item[T060] Einen Aspekt aus der Liste auswählen
	
	\item[T070] Scrollen
	
	\item[T080] Schieberegler einstellen
	
	\item[T090] Zur Karte zurückkehren
	
	\item[T100] Seite schließen
\end{Kriterien}
\subsubsection{Erweiterte Testfälle}
\begin{Kriterien}{TF}
	
	\item Zwischen verschiedenen Kartenansichten wechseln
	
\end{Kriterien}
\subsubsection{Stabilitätstests}
\begin{Kriterien}{MK}

	\item Viele Daten gleichzeitig anfordern

	\item Schnelles Anfordern von daten
	
\end{Kriterien}
