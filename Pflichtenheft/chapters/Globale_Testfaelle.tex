\section{Globale Testfälle}
\setcounter{counter}{10}
\subsection{Basis-Testfälle}
\begin{Kriterien}{TF}

	\item[Webanwendung öffnen] Der Benutzer startet die \gls{Webanwendung}. \\ \textbf{Erwartetes Ergebnis:} Startseite mit einer Kartenansicht öffnet sich.

	\item[Karte bewegen] Der Benutzer zieht mit dem Eingabegerät an der Karte. \textbf{Erwartetes Ergebnis:} \\ Die Karte bewegt sich.
	
	\item[Zoomen] Der Benutzer führt mit den Fingern eine Zoom-Geste aus oder Scrollt mit dem Mausrad \\ \textbf{Erwartetes Ergebnis:} Die Karte zoomt.
	
	\item[Einen Pin einer Messtation anklicken] Der Benutzer klickt einen Pin einer Messtation auf der Karte an. \\ \textbf{Erwartetes Ergebnis:} Das Stationen Pop-Up erscheint.
	
	\item[Karte auswählen] Der Benutzer wählt in der Liste mit einem Knopfdruck aus, welche Ansicht die Karte anzeigen soll. \\ \textbf{Erwartetes Ergebnis:} Die Karte wechselt zu dieser Ansicht.
	
	\item[Scrollen] Der Benutzer scrollt in der Detailansicht runter. \\ \textbf{Erwartetes Ergebnis:} Die Seitenansicht bewegt sich nach unten und die verschiedenen Diagramme werden angezeigt.
	
	\item[Ort suchen] Der Benutzer tippt einen Straßennamen in die Suchmaschine und klickt Enter. \\ \textbf{Erwartetes Ergebnis:} Die Karte springt auf die Straße mit dem gesuchten Namen.
	
	\item[Fehlermeldung bei der Suche] Der Benutzer gibt einen nicht-existierenden Ort ein. \\ \textbf{Erwartetes Ergebnis:} Pop-Up Fenster öffnet sich mit einer Fehlermeldung.
	
	\item[Zur Karte zurückkehren] Der Benutzer drückt den "Zurück zur Karte" Button. \\ \textbf{Erwartetes Ergebnis:} Die Seite mit der Karte wird geöffnet. 
	
	\item[Webanwendung schließen] Der Benutzer schließt die Webanwendung im Browser. \\ \textbf{Erwartetes Ergebnis:} Webanwendung wird geschlossen.
\end{Kriterien}
\subsection{Erweiterte Testfälle}
\begin{Kriterien}{TF}
	
	\item[Sprache wechseln] Der Benutzer klickt eine Flaggen-Icon an. \\ \textbf{Erwartetes Ergebnis:} Die Webanwendung wird in der gewählten Sprache angezeigt.
	
	\item[An Städte einzoomen] Der Benutzer klickt an ein Polygon in der verkleinerten Karte. \\ \textbf{Erwartetes Ergebnis:} Die Karte zoomt ein und die \glspl{Station} in der Stadt werden markiert.
	
	\item[Durchschnittsdiagramm] Der Benutzer drückt bei der Kartenansicht einer Stadt den Button für das Durchschnittsdiagramm. \\ \textbf{Erwartetes Ergebnis:} Ein Bubble mit dem Durchschnittsdiagramm öffnet sich. Dabei werden die \glspl{Graph} von allen Jahren mit verfügbaren \glspl{Messwert}en angezeigt, sowie der allgemeiner Durchschnittswert.
	
	\item[Sensorinformationen] Der Benutzer klickt in der Detailansicht den Index-Button an. \\ Erwartetes Ergebnis: Ein Pop-Up Fenster
wird geöffnet mit den Sensorinformationen.

	\item[Zum jetzigen Standort springen] Der Benutzer erlaubt Standortermittlung und klickt auf den Button "Zu meinem Standort springen". \\ Erwartetes Ergebnis: Die Karte springt auf den aktuellen Standort des Nutzers und zoomt ein.
	
	\item[Position der Karte merken 1] Der Nutzer wechselt welchen Feature die Karte anzeigen soll. \\ 
	\textbf{Erwartetes Ergebnis:} Die gewechselte Karte bleibt in der gleichen Position wie vor dem Wechsel.
	
	\item[Position der Karte merken 2] Der Nutzer kehrt vom Detailansicht zurück auf die Karte. \\ 
	\textbf{Erwartetes Ergebnis:} Die Karte ist an der gleichen Position wie vor dem Öffnen der Detailansicht.
	
\end{Kriterien}
\subsection{Stabilitättests}
\begin{Kriterien}{TF}

	\item[Viele Daten gleichzeitig anfordern] Der Nutzer zoomt so weit aus wie möglich.\\ \textbf{Erwartetes Ergebnis:} Alle Städte werden mit farbigen Polygonen markiert.

	\item[Schnelles Anfordern der Daten] Der Nutzer wechselt schnell zwischen Detailansicht und Kartenansicht, wobei er immer andere \glspl{Station} anklickt. \\ \textbf{Erwartetes Ergebnis:} Alle Karten und Diagramm werden geladen.

\end{Kriterien}
