\section{Produkteinsatz}

\subsection{Anwendungsbereiche}

Die \gls{Webanwendung} dient dafür dem Benutzer Informationen der \gls{SmartAQnet} Datenbank oder eines anderen \gls{FROST-Server} benutzerfreundlich darstellen zu lassen.

\subsection{Zielgruppen}

Benutzer der \gls{Webanwendung} sind interessierte Bürgerinnen und Bürger, die sich über Luftqualitätsdaten informieren möchten. 
Dabei wird die \gls{Webanwendung} einfach gehalten, um älteren Menschen, die im Umgang mit informationstechnischen Systemen noch nicht sehr erfahren sind, die Benutzung zu erleichtern.
Ob die Benutzer schon Kenntnisse im Bereich der Luftqualität haben, soll für die Benutzung der \gls{Webanwendung} unerheblich sein.

\subsection{Betriebsbedingungen}

Die \gls{Webanwendung} sollte rund um die Uhr und unbeaufsichtigt für prinzipiell unbeschränkt viele Benutzer zur Verfügung stehen, sofern der \gls{Webserver} auf dem die \gls{Webanwendung} läuft dies zulässt. 
Vorrausgesetzt für den Betrieb der \gls{Webanwendung} ist der Zugriff auf einen \gls{FROST-Server}.